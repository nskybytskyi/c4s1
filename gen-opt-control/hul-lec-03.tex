intro + recap

\subsubsection{Початкові значення інтегралу}

Дослідимо, за яких умов початкове значення інтегралу ($(I_0^\alpha f)(t) = o(1)$) дорівнює нулю.

\begin{theorem}
    Нехай $\alpha > 0$, $p > 1 / \alpha$, $p \ge 1$, $f \in L_p((0, T))$. Тоді $(I_0^\alpha) (t) = o(t^{\alpha - 1 / p})$ при $t \to 0$.
\end{theorem}

\begin{proof}
    \begin{equation}
        \begin{aligned}
            \left| (I_0^\alpha) (t) \right| &= \frac{1}{\Gamma(\alpha)} \left| \int_0^t f(s) (t - s)^{\alpha - 1} \diff s \right| \le \\
            &\le \frac{1}{\Gamma(\alpha)}  \int_0^t \left| f(s) (t - s)^{\alpha - 1} \diff s \right| \le \\
            &\le \frac{1}{\Gamma(\alpha)} \left( \int_0^t |f(s)|^p \diff s \right)^{1/p} \left( \int_0^t (t - s)^{(\alpha - 1) q} \diff s \right)^{1/q} = \\
            &= \frac{1}{\Gamma(\alpha)} \left( \int_0^t |f(s)|^p \diff s \right)^{1/p} \left( \frac{t^{(\alpha - 1) q + 1}}{(\alpha - 1) q + 1} \right)^{1 / q} = \\
            &= \left( \int_0^t |f(s)|^p \diff s \right)^{1/p} \left( \frac{t^{\alpha - 1 + 1 / q}}{c(\alpha, p)} = \\
            &= \left( \int_0^t |f(s)|^p \diff s \right)^{1/p} \left( \frac{t^{\alpha - 1 / p}}{c(\alpha, p)} = \\
            &= o(t^{\alpha - 1 / p}),
        \end{aligned}
        де останній перехід справджується адже $\int_0^t |f(s)|^p \diff s = o(1)$ при $t \to 0$
    \end{equation}

    \begin{remark}[абсолютна неперервність інтеграла Лебега]
        Якщо $f \in L_1$ то $\forall \epsilon > 0$ $\exists \delta(\epsilon) > 0$: $|forall $
    \end{remark}

    \begin{remark}[інтегральна нерівність Коші-Буняковського]
        \begin{equation}
            \|f g\|_{L_1} \le \|f\|_{L_p} \|g\|_{L_q},  
        \end{equation}
        де $1 / p + 1 / q = 1$.
    \end{remark}

    \begin{remark}[інтегральна нерівність Гельдера]
        \begin{equation}
            \|f g\|_{L_1} \le \|f\|_{L_p} \|g\|_{L_q},  
        \end{equation}
        де $1 / p + 1 / q = 1$.
    \end{remark}

    \begin{remark}
        Умова $p > 1 / \alpha$ необхідна для збіжності усіх інтегралів з доведення
    \end{remark}
\end{proof}

\begin{corollary}
    При $\alpha > 1 / p$ маємо $(I_0^\alpha f)(t) = o(1)$, тобто $(I_0^\alpha f)(0) = 0$.
\end{corollary}

\begin{exercise}
    Наведіть приклад $f$ для якої $(I_0^\alpha f)(0) \ne 0$ (але і не $\infty$).
\end{exercise}

\subsubsection{Початкові значення похідних}

\begin{theorem}
    Нехай $\alpha > 0$, $\alpha \not\in \NN$, $n = \lceil \alpha \rceil$, $f \in C^{n - 1}([0, T])$, $p > \frac{1}{n - \alpha}$, $f^{(n)} \in L_P([0, T])$. Тоді $(D_0^\alpha) (0) = 0 \iff f^{(k)} (0) = 0$ при $k = \overline{0, n - 1}$.
\end{theorem}

\begin{proof}
    За умов теореми
    \begin{equation}
        \label{eq:3-to-ref}
        (D_0^\alpha f)(t) = \frac{1}{\Gamma(n - \alpha)} \int_0^t \frac{f^{(n)}^s}{(t - s)^{\alpha - n + 1}} \diff s + \sum_{k = 0}^{n - 1} \frac{f^{(k)}(0) t^{k - \alpha}}{\Gamma(k - \alpha + 1)}.
    \end{equation}

    ($\Longleftarrow$) У формулі вище інтеграл дорівнює нулю за першою сьогоднішньою теоремою, а уся сумма зануляється за умовою теореми. \medskip

    ($\Longrightarrow$) Домножатимемо \eqref{eq:3-to-ref} на $t^{\alpha - k}$ для $k = \overline{0, n - 1}$. Наприклад, для $k = 0$ матимемо
    \begin{multline}
        t^\alpha (D_0^\alpha f)(t) = t^\alpha ({}^\star D_0^\alpha f)(t) + \\
        + \frac{f(0)}{\Gamma(1 - \alpha)} + \sum_{k = 1}^{n - 1} \frac{f^{(k)}(0) t^k}{\Gamma(k - \alpha + 1)}.
    \end{multline}

    Бачимо, що $t^\alpha ({}^\star D_0^\alpha f)(t) = o(1)$, всі доданки суми нескінченно малі, тому $f(0) = 0$. Далі за індукцією по $k$ отримуємо рівність нулеві усіх похідних до $(n-1)$-ої.
\end{proof}

