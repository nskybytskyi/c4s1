\subsection{Властивості дробових похідних}

\begin{example}
    Знайдемо похідні степеневих функцій. Нехай $\beta > -1$, $0 < \alpha < 1$. Тоді
    \begin{equation}
        D_0^\alpha t^\beta = \frac{\diff}{\diff t} I_0^{1-\alpha} t^\beta.
    \end{equation}
    \begin{equation}
        I_0^{1-\alpha} t^\beta = \frac{1}{\Gamma(1 - \alpha)} \int_0^t s^\beta (t- s)^{-\alpha} \diff s
    \end{equation}
    Нагадаємо означення бета-функції:
    \begin{equation}
        B(a,b) = \int_0^1 \nu^{a-1}(1-\nu)^{b-1}.
    \end{equation}
    Проведемо заміну $\nu = s / t$, тоді $\diff s = t \diff \nu$, отримаємо
    \begin{equation}
        \frac{1}{\Gamma(1-\alpha)} \int_0^1 (t \nu)^\beta t^{-\alpha} (1 - \nu) t \diff \nu = \frac{t^{\beta - \alpha + 1}}{\Gamma(1 - \alpha)} B(\beta + 1, 1 - \alpha) = \frac{\Gamma(\beta + 1)}{\Gamma(2 + \beta - \alpha)} t^{\beta - \alpha + 1}.
    \end{equation}
    Лишилося продиференційювати цей інтеграл:
    \begin{equation}
        D_0^\alpha \frac{\Gamma(\beta + 1)}{\Gamma(2 + \beta - \alpha)} t^{\beta - \alpha + 1} = \frac{\Gamma(\beta + 1)}{\Gamma(2 + \beta - \alpha)} (\beta - \alpha + 1) t^{\beta - \alpha} = \frac{\Gamma(\beta + 1)}{\Gamma(1 + \beta - \alpha)} t^{\beta - \alpha},
    \end{equation}
    де ми скористалився властивістю $a \Gamma(a) = \Gamma(a + 1)$.
\end{example}
\begin{remark}
    Ця формула справедлива і для $\alpha \ge 1$, але умова $\beta > -1$ важлива для збіжності 
\end{remark}

\begin{example}
    Зокрема, якщо $\alpha \le \beta \in \NN$, то маємо формулу
    \begin{equation}
        \frac{\diff^\alpha}{\diff t^\alpha} t^\beta = \frac{\beta!}{(\beta - \alhpa)!} t^{\beta - \alpha}.
    \end{equation}
    Наприклад
    \begin{equation}
        \frac{\diff^2}{\diff t^2} t^4 = \frac{4!}{2!} t^2.
    \end{equation}
\end{example}

Зрозуміло також що всі введені нами оператори лінійні. \medskip

\begin{proposition}
    На жаль, не виконується наступна властивість
    \begin{equation}
        \frac{\diff^n}{\diff t^n} e^{\lambda t} = \lambda^n e^{\lambda t}.
    \end{equation}
\end{proposition}

\begin{proof}
    \begin{equation}
        e^{\lambda t} = \sum_{k = 0}^\infty \frac{(\lambda t)^k}{k!}.
    \end{equation}
    Почленно диференціюємо:
    \begin{equation}
        \begin{aligned} 
            D_0^\alpha e^{\lambda t} 
            &= D_0^\alpha \left( \sum_{k = 0}^\infty \frac{(\lambda t)^k}{k!} \right) = \\
            &= \sum_{k = 0}^\infty \frac{\lambda^k}{k!} D_0^\alpha \left( t^k \right) = \\
            &= \sum_{k = 0}^\infty \frac{\lambda^k}{k!} \frac{k!}{\Gamma(k + 1 - \alpha} t^{k - \alpha} \ne \\
            &\ne \sum_{k = 0}^\infty \lambda^\alpha \frac{(\lambda t)^k}{k!}.
        \end{aligned}
    \end{equation}
\end{proof}

\begin{proposition}[(не) формула Ньютона-Лейбніца]
    \begin{equation}
        \frac{\diff}{\diff t} \int_0^t f(s) \diff s
    \end{equation}
\end{proposition}

\begin{proposition}[формула Ньютона-Лейбніца]
    \begin{equation}
        \int_0^t f'(s) \diff s = f(t) - f(0).
    \end{equation}
\end{proposition}

\begin{proposition}[напівгрупова властивість дробових інтегралів]
    Нехай $\alpha, \beta > 0$, тоді $I_0^{\alpha + \beta} = I_0^\alpha I_0^\beta$.
\end{proposition}
\begin{exercise}
    Доведіть цю властивість.
\end{exercise}
\begin{proof}
    За означенням, $I_0^\alpha f = f \star y_\alpha$, тому достатньо перевірити $y_{\alpha + \beta} = y_\alpha \star y_\beta$.
\end{proof}

\begin{theorem}[Аналог (не) формули Ньютона-Лейбніца]
    Для $\alpha > 0$
    \begin{equation}
        D_0^\alpha I_0^\alpha f = f.
    \end{equation}
\end{theorem}
\begin{proof}
    Нехай $n = \lceil \alpha \rceil$, тоді
    \begin{equation}
        D_0^\alpha I_0^\alpha f = \frac{\diff^n}{\diff t^n} I_0^{n - \alpha} I_0^{\alpha} f = \frac{\diff^n}{\diff t^n} I_0^n f = f. 
    \end{equation}
    \begin{remark}
        Тут ми скористалися напівгруповою властивістю.
    \end{remark}
\end{proof}

\begin{proposition}[Аналог формули Ньютона-Лейбніца]
    Нехай $f, D_0^\alpha f \in L_1([0,T])$, $n = \lceil \alpha \rceil$, $\alpha \not\in \NN$, тоді для $0 < t < T$ маємо
    \begin{equation}
        (I_0^\alpha D_0^\alpha f)(t) = f(t) - \sum_{k = 0}^{n - 1} (D_0^{\alpha - k  1} f)(0) \cdot \frac{t^{\alpha - k - 1}}{\Gamma(\alpha - k}.
    \end{equation}
\end{proposition}
\begin{remark}
    Тут під $D_0^{-|\beta|}$ маємо на увазі $I_0^{|\beta|}$.
\end{remark}
\begin{example}
    Для $0 < \alpha < 1$ маємо
    \begin{equation}
        (I_0^\alpha D_0^\alpha f)(t) = f(t) - (I_0^{1 - \alpha} f)(0) \frac{t^{\alpha - 1}}{\Gamma(\alpha)}.
    \end{equation}
\end{example}
\begin{remark}
    Тут $(I_0^{1 - \alpha} f)(0) = \lim_{\epsilon \to +0} (I_0^{1 - \alpha} f)(\epsilon)$.
\end{remark}
\begin{proof}
    Доведемо частинний випадок:
    \begin{equation}
        \begin{aligned}
            (I_0^\alpha D_0^\alpha f)(t) 
            &= \frac{1}{\Gamma(\alpha)} \int_0^t (t - s)^{\alpha - 1} (D_0^\alpha f) (s) \diff s = \\
            &= \frac{\diff}{\diff t} \left( \frac{1}{\alpha \Gamma(\alpha)} \int_0^t (t - s)^{\alpha} (D_0^\alpha f) (s) \diff s \right).
        \end{aligned}
    \end{equation}
    Виконаємо наступні маніпуляції з виразом що стоїть під похідною:
    \begin{equation}
        \begin{aligned}
            \frac{1}{\alpha \Gamma(\alpha)} \int_0^t (t - s)^{\alpha} (D_0^\alpha f) (t) \diff t
            &= \frac{1}{\alpha \Gamma(\alpha)} \left( \left. (t - s)^{\alpha} I_0^{1-\alpha} f (s) \right|_{s = 0}^{s = t} + \alpha \int_0^t (t - s)^{\alpha - 1} I_0^{1-\alpha} f(s) \diff s \right) = \\
            &= - \frac{t^\alpha (I_0^{1 - \alpha} f)(0)}{\alpha \Gamma(\alpha)} + I_0^\alpha I_0^{1-\alpha} f = \\
            &= - \frac{t^\alpha (I_0^{1 - \alpha} f)(0)}{\alpha \Gamma(\alpha)} + I_0^1 f.
        \end{aligned}
    \end{equation}
    Лишилося всього лише продиференціювати:
    \begin{equation}
        \frac{\diff}{\diff t} \left( - \frac{t^\alpha (I_0^{1 - \alpha} f)(0)}{\alpha \Gamma(\alpha)} + I_0^1 f \right) = f(t) - \frac{t^{\alpha - 1} (I_0^{1-\alpha} f)(0)}{\Gamma(\alpha)}.
    \end{equation}
\end{proof}

Для похідних за Капуто:
\begin{theorem}
    Нехай $f \in L_\infty([0,T])$, тобто $\exists M \in \RR$: $|f(t)| \overset{\text{a.e.}}{\le} M$, тоді $({}^\star D_0^\alpha I_0^\alpha f)($
\end{theorem}
а також
\begin{theorem}
    Нехай $n = \lceil \alpha \rceil$, $f \in AC^n([0,T])$, тоді
    \begin{equation}
        (I_0^\alpha {}^\star D_0^\alpha f)(t) = f(t) - \sum_{k = 0}^n \frac{f^{(k)}(0)}{k!} t^k.
    \end{equation}
\end{theorem}
\begin{remark}
    Ця формула справедлива і для цілих $\alpha$
\end{remark}

\begin{proposition}
    Для похідних у загальному випадку не виконується напівгрупова властивість.
\end{proposition}
\begin{theorem}
    Нехай $f, D_0^\beta \in L_1([0, T])$, $\alpha \not\in \NN$. Тоді
    \begin{equation}
        (D_0^\alpha D_0^\beta f)(t) = (D_0^{\alpha + \beta}) f(t) - \sum_{k = 0}^{\lceil \beta \rceil - 1} (D_0^{\beta - k - 1} f)(0) \frac{t^{-\alpha - k - 1}}{\Gamma(-\alpha - k)}
    \end{equation}
\end{theorem}
\begin{example}
    Зокрема для $0 < \alpha, \beta < 0$
    \begin{equation}
        (D_0^\alpha D_0^\beta f)(t) = (D_0^{\alpha + \beta}) f(t) - (I_0^{1 - \beta} f)(0) \frac{t^{-\alpha - 1}}{\Gamma(-\alpha)}
    \end{equation}
\end{example}
\begin{proof}
    Доведемо частинний випадок:
    \begin{equation}
        \begin{aligned}
            (D_0^\alpha D_0^\beta f)(t)
            &= \left( \frac{\diff}{\diff t} I_0^{1 - \alpha} D_0^\beta f \right)(t) = \\
            &= \left( \frac{\diff^2}{\diff t^2} I_0^{2 - \alpha} D_0^\beta f \right)(t) = \\
            &= \left( \frac{\diff^2}{\diff t^2} I_0^{2 - \alpha - \beta} I_0^\beta D_0^\beta f \right)(t) = \\
            &= \frac{\diff^2}{\diff t^2} I_0^{2 - \alpha - \beta} \left( f(t) - \frac{(I_0^{1-\beta} f)(0) t^{\beta - 1}}{\Gamma(\beta)} \right) = \\
            &= (D_0^{\alpha + \beta} f)(t) - \frac{(I_0^{1-\beta} f)(0)}{\Gamma(\beta)} D_0^{\alpha+\beta} t^{\beta-1} = \\
            &= (D_0^{\alpha + \beta} f)(t) - \frac{(I_0^{1-\beta} f)(0)}{\Gamma(\beta)} \frac{\Gamma(\beta)}{\Gamma(-\alpha)} t^{-1-\alpha} = \\
            &= (D_0^{\alpha + \beta} f)(t) - \frac{(I_0^{1-\beta} f)(0)}{\Gamma(-\alpha)} t^{-1-\alpha}.
        \end{aligned}
    \end{equation}
\end{proof}
