\begin{lemma}
    $\mathscr{L}[c](\eta) = c / \eta$.
\end{lemma}

\begin{lemma}
    $\mathscr{L}[e^{p t} f(t)](\eta) = \mathscr{L}[f(t)](\eta - p)$.
\end{lemma}

Нагадаємо, що раніше ми з'ясували, що
$\mathscr{L}[t^{-\beta}](\eta) = \Gamma(1 - \beta) \eta^{\beta - 1}$.

\begin{theorem}[Таубера]
    Нехай $- \beta > - 1$, $f$ монотонна при великих $t$ (тобто вона монотонна на деякому проміжку вигляду $[t_0,+\infty)$). Тоді $f(t) \sim t^{-\beta}$ при $t \to +\infty$ $\iff$ $\mathscr{L}[f](\eta) = \Gamma(1 - \beta) \eta^{\beta - 1}$ при $\eta \to 0$.
\end{theorem}

\begin{remark}
    Тут $f(x) \sim g(x)$ при $x \to \infty$ означає, що $\lim_{x \to \infty} \frac{f(x)}{g(x)} = 1$.
\end{remark}

\begin{corollary}
    Нехай $0 < \alpha < 1$ і $f$ монотонна при великих $t$, $f \ge 0$ на $[0, +\infty)$ і $\int_0^{+\infty} f(t) \diff t = 1$. Тоді $\forall A > 0$: $f(t) \sim \alpha A t^{-\alpha - 1}$ при $t \to +\infty$ $\iff$ $\mathscr{L}[f](\eta) = 1 - A \Gamma(1 - \alpha) \eta^\alpha + o(n^\alpha)$ при $\eta \to 0+$.
\end{corollary}

\begin{proof}
    Розглянемо функцію 
    \begin{equation}
        F(t) = \int_t^{+\infty} f(s) \diff s.
    \end{equation}
    Зауважимо, що $F'(t) = -f(t)$.

    \begin{exercise}
        Доведіть, що за наших припущень
        \begin{equation}
            f(t) \sim A \alpha t^{-\alpha - 1} \iff F(t) \sim A t^{-\alpha}.
        \end{equation}
    \end{exercise}
    \begin{proof}
        ($\Longrightarrow$) Якщо $f(t) \sim A \alpha t^{-\alpha - 1}$, то, за визначенням асимптотики, 
            \begin{equation}
                (\forall \epsilon > 0) (\exists t_0(\epsilon)) (\forall t > t_0) \quad \left| \frac{f(t)}{A \alpha t^{-\alpha - 1}} - 1 \right| < \epsilon,
            \end{equation}
            або ж, що те саме,
            \begin{equation}
                (\forall \epsilon > 0) (\exists t_0(\epsilon)) (\forall t > t_0) \quad (A - \epsilon) \alpha t^{-\alpha - 1} < f(t) < (A + \epsilon) \alpha t^{-\alpha - 1},
            \end{equation}
            щоправде вже з іншим $t_0(\epsilon)$, але не суть. \medskip
            
            Інтегруємо:
            \begin{multline}
                (\forall \epsilon > 0) (\exists t_0(\epsilon)) (\forall t > t_0) \\ \int_t^\infty (A - \epsilon) \alpha s^{-\alpha - 1} \diff s < \int_t^\infty f(s) \diff s < \int_t^\infty (A + \epsilon) \alpha s^{-\alpha - 1} \diff s,
            \end{multline}
            звідки
            \begin{equation}
                (\forall \epsilon > 0) (\exists t_0(\epsilon)) (\forall t > t_0) \quad (A - \epsilon) t^{-\alpha} < F(t) < (A + \epsilon) t^{-\alpha},
            \end{equation}
            отримали що хотіли. \medskip
            
        ($\Longleftarrow$) Припустимо, що $F(t) \sim A t^{-\alpha}$, але $f(t) \not\sim A \alpha t^{-\alpha - 1}$. За визначенням, це означає, що  
        \begin{equation}
            (\exists \epsilon > 0) (\forall t_0) (\exists t > t_0) \quad \left| \frac{f(t)}{A \alpha t^{-\alpha - 1}} - 1 \right| > \epsilon.
        \end{equation}
        
        Зрозуміло, що для деякого $C > 0$ нескінченно часто відбувається або
        \begin{equation}
            \frac{f(t)}{A \alpha t^{-\alpha - 1}} > 1 + C,
        \end{equation}
        або
        \begin{equation}
            \frac{f(t)}{A \alpha t^{-\alpha - 1}} < 1 - C,
        \end{equation}
        а також нескінченно часто відбувається
        \begin{equation}
            \frac{f(t)}{A \alpha t^{-\alpha - 1}} = 1.
        \end{equation}
        
        Без обмеження загальності, перше, розглянемо тоді зростаючу $\{t_n\}_{n = 1}^\infty$ таку, що $t_n \to \infty$ при $n \to \infty$ і 
        \begin{equation}
            \frac{f(t_n)}{A \alpha t_n^{-\alpha - 1}} = 1.
        \end{equation}
        
        Позначимо
        \begin{equation}
            T_n = \min_{t \ge t_n} \left\{t: \frac{f(t)}{A \alpha t^{-\alpha - 1}} = 1 + C \right\}.
        \end{equation}
        
        Для зручності запишемо це як 
        \begin{equation}
            f(T_n) = (1 + C) A \alpha T_n^{-\alpha - 1}.
        \end{equation}
        
        Без обмеження загальності вважаємо, що $f$ монотонна починаючи з деякого $t_0 < t_1$, а тому $f(T_n) \le f(t_n)$. Тоді
        \begin{equation}
            (1 + C) A \alpha T_n^{-\alpha - 1} \le A \alpha t_n^{-\alpha - 1}.
        \end{equation}
        
        Логарифмуючи маємо
        \begin{equation}
            \ln (1 + C) + \ln A + \ln \alpha - (\alpha + 1) \ln T_n \le \ln A + \ln \alpha - (\alpha + 1) \ln t_n,
        \end{equation}
        або ж
        \begin{equation}
            (\alpha + 1) \ln T_n \ge (\alpha + 1) \ln t_n + \ln (1 + C).
        \end{equation}
        
        Звідси
        \begin{equation}
            \ln T_n \ge \ln t_n + k,
        \end{equation}
        або ж
        \begin{equation}
            T_n \ge K t_n,
        \end{equation}
        де $k = \frac{\ln (1 + C)}{\alpha + 1}$ --- додатнє, $K = e^k$. Аналогічним чином можна показати, що $f(t) \ge (1 + C') A \alpha t^{-\alpha - 1}$ на $[T_n/K', T_n]$ для певних сталих $C' < C$ і $1 < K' < K$. \medskip
        
        Розглянемо тепер $\int_{t_n}^{T_n} f(s) \diff s$. Доволі просто показати, що 
        \begin{equation}
            \begin{aligned}
                \int_{T_n / K'}^{T_n} f(s) \diff s 
                &< (A - \epsilon) (T_n/K')^{-\alpha} - (A + \epsilon) T_n^{-\alpha} = \\
                &= A \Big( (T_n/K')^{-\alpha} - T_n^{-\alpha} \Big) - \epsilon \Big( (T_n/K')^{-\alpha} + T_n^{-\alpha} \Big).
            \end{aligned}
        \end{equation}
        для довільного $\epsilon > 0$, починаючи з деякого $n_0(\epsilon)$, звичайно. Але ж $f(t) \ge (1 + C') A \alpha t^{-\alpha - 1}$ на $[T_n/K', T_n]$, а тому
        \begin{equation}
            \begin{aligned}
                \int_{T_n/K'}^{T_n} f(s) \diff s 
                &\ge \int_{T_n/K'}^{T_n} (1 + C') A \alpha t^{-\alpha - 1} = \\
                &= (1 + C') A (T_n/K')^{-\alpha} - (1 + C') A T_n^{-\alpha} = \\
                &= A \Big( (T_n/K')^{-\alpha} - T_n^{-\alpha} \Big) + C' A \Big( (T_n/K')^{-\alpha} - T_n^{-\alpha} \Big).
            \end{aligned}
        \end{equation}
        
        Лишилося порівняти (якщо права частина більше то ми отримали протиріччя)
        \begin{equation}
            \epsilon \Big( (T_n/K')^{-\alpha} + T_n^{-\alpha} \Big) \lor C' A \Big( (T_n/K')^{-\alpha} - T_n^{-\alpha} \Big),
        \end{equation}
        або ж
        \begin{equation}
            \epsilon T_n^{-\alpha} K_2 \lor C' A T_n^{-\alpha} K_3,
        \end{equation}
        де $K_2 = (K')^\alpha + 1$, $K_3 = (K')^\alpha - 1$ --- додатні константи. \medskip
        
        Помітимо, що тепер $T_n^{-\alpha}$ можна скоротити, отримаємо
        \begin{equation}
            \epsilon K_2 \lor C' A K_3.
        \end{equation}
        
        Цілком очевидно, що при $\epsilon \to 0$ права частина переважає, а тому отримали протиріччя.
    \end{proof}

    Розглянемо
    \begin{equation}
        \mathscr{L}[F](\eta) = \int_0^{+\infty} e^{-\eta t} \int_t^{+\infty} f(s) \diff s \diff t,
    \end{equation}
    % картинка Тоді, з
    Змінюючи порядок інтегрування, отримуємо
    \begin{equation}
        \begin{aligned}
            \int_0^{+\infty} f(s) \int_0^s e^{-\eta t} \diff t \diff s
            &= \int_0^{+\infty} f(s) \frac{1 - e^{-\eta s}}{\eta} \diff s = \\
            &= \frac{1}{\eta} \left( 1 - \int_0^{+\infty} f(s) e^{-\eta s} \diff s \right) = \\
            &= \frac{1 - \mathscr{L}[f](\eta)}{\eta}.
        \end{aligned}
    \end{equation}

    Звідси
    \begin{equation}
        \mathscr{L}[f](\eta) = 1 - \eta \mathscr{L}[F](\eta).
    \end{equation}

    Тепер можемо записати
    \begin{equation}
        \begin{aligned}
            f(t) \underset{t \to \infty}{\sim} A \alpha t^{-\alpha - 1}
            &\iff F(t) \underset{t \to \infty}{\sim} A t^{-\alpha} \iff \\
            &\iff \mathscr{L}[F](\eta) \underset{\eta \to 0+}{\sim} A \Gamma(1 - \alpha) \eta^{\alpha - 1} \iff \\
            &\iff \mathscr{L}[F](\eta) = \Gamma(1 - \alpha) \eta^{\alpha - 1} + o(\eta^{\alpha - 1}) \iff \\
            &\iff \mathscr{L}[f](\eta) = 1 - A \Gamma(1 - \alpha) \eta^{\alpha} + o(\eta^{\alpha}).
        \end{aligned}
    \end{equation}
\end{proof}

\section{Моделі аномальної дифузії}

% мотивація

Нагадаємо, що класичні рівняння дифузії та теплопровідності зумовлені наступними чинниками:
\begin{enumerate}
    \item закон збереження кількості речовини/тепла;
    \item джерела і стоки;
    \item ``закон'' Фіка/Фур'є --- емпірично встановлене для широкого класу процесів твердження про те, що інтенсивність потоку речовини/тепла пропорційна мінус градієнту концентрації речовини/кількості тепла на границі.
\end{enumerate}

Розглянемо тепер інший підхід, який грунтується на випадкових блуканнях з неперервним часом (eng. \textit{CTRW, continuous time random walk}). А саме, нехай $x(t)$ --- випадкова величина (координата частинки в момент часу $t$), а  $u(x, t)$ (gпри фіксованому $t$ --- щільність координати частинки). \medskip

% картинка

Параметри моделі:
\begin{enumerate}
    \item $u_0(x)$ --- щільність початкового (при $t = 0$) розподілу;
    \item $\psi(t)$ --- щільність часу очікування наступного стрибка;
    \item $\lambda(x)$ --- щільність зміщення.
\end{enumerate}

\begin{definition}
    Нехай $f: \RR \to \RR$, тоді її \textit{перетворенням Фур'є} називається 
    \begin{equation}
        \mathcal{F}[f](\omega) = \tilde{f}(\omega) = \int_\RR e^{i \omega t} f(x) \diff x.
    \end{equation}
\end{definition}

\begin{proposition}
    Для перетворення Фур'є справедлива теоерма згортки:
    \begin{equation}
        \mathcal{F}[f * g](\omega) = \mathcal{F}[f](\omega) \cdot \mathcal{F}[f](\omega),
    \end{equation}
    де
    \begin{equation}
        (f * g)(x) = \int_\RR f(y) g(x - y) \diff y.
    \end{equation}
\end{proposition}

\begin{definition}
    Нехай $u(x, t): \RR \times \RR \to \RR$, тоді її \textit{перетворенням Фур'є-Лапласа} називається 
    \begin{equation}
        \mathcal{F}\text{-}\mathscr{L}[u](\omega, \eta) = \tilde{\bar{u}}(\omega, \eta) = \int_0^{+\infty} \int_\RR e^{-\eta t + i \omega x} u(x, t) \diff x \diff t.
    \end{equation}
\end{definition}

\begin{th_formula}[Монтрола-Вайса]
    \begin{equation}
        \mathcal{F}\text{-}\mathscr{L}[u](\omega, \eta) = \frac{\mathcal{F}[u_0](\omega)}{\eta} \frac{1 - \mathscr{L}[\psi](\eta)}{1 - \mathscr{L}[\psi](\eta) \mathcal{F}[\lambda](\omega)},
    \end{equation}
    як тільки $|\mathscr{L}[\psi](\eta) \mathcal{F}[\lambda](\omega)| < 1$.
\end{th_formula}
\begin{proof}
    Введемо додаткове позначення: $n(t)$ --- кількість стрибків до моменту $t$. $\psi_k(t)$ --- щільність часу $k$-го стрибка. І нарешті, $\lambda_k(x)$ --- щільність координати після $k$-го стрибка. \medskip

    Запишемо формулу повної ймовірності:
    \begin{equation}
        \begin{aligned}
            u(x, t)
            &= \sum_{k = 0}^\infty \mathsf{P}\{n(t) = k\} \lambda_k (x) = \\
            &= \sum_{k = 0}^\infty \Big(\mathsf{P}\{n(t) \ge k\} - \mathsf{P}\{n(t) \ge k + 1\}\Big) \lambda_k (x) = \\
            &= \sum_{k = 0}^\infty \left( \int_0^t \psi_k(s) \diff s - \int_0^t \psi_{k + 1} (s) \diff s \right) \lambda_k (x) = \\
            &= \sum_{k = 0}^\infty \left( \int_0^t \psi^{\star k}(s) \diff s - \int_0^t \psi^{\star(k + 1)} (s) \diff s \right) \cdot (u_0 * \lambda^{*k}) (x).
        \end{aligned}
    \end{equation}

    \begin{remark}
        Тут ми скористалися тим, що $\psi_k(x) = \psi^{\star k} = \underset{k}{\underbrace{\psi \star \psi \star \ldots \star \psi}}$ і $\lambda_k(x) = u_0 * \lambda^{* k} = u_0 * \underset{k}{\underbrace{\lambda * \lambda * \ldots * \lambda}}$.
    \end{remark}
    
    \begin{exercise}
        Поновити у конспекті формальне доведення цих фактів.
    \end{exercise}

    З урахуванням \%перелік\_нагадувань\%, маємо
    \begin{equation}
        \begin{aligned}
            \mathcal{F}\text{-}\mathscr{L}[u](\omega, \eta) 
            &= \sum_{k = 0}^\infty \frac{1}{\eta} \Big( (\mathscr{L}[\psi](\eta)^k - \mathscr{L}[\psi](\eta)^{k + 1}) \mathcal{F}[u_0](\omega) \mathcal{F}[\lambda](\omega)^k \Big) = \\
            &= \frac{\mathcal{F}[u_0](\omega) (1 - \mathscr{L}[\psi](\eta))}{\eta} \sum_{k = 0}^\infty \mathscr{L}[\psi](\eta)^{k} \mathcal{F}[\lambda](\omega)^k = \\
            &= \frac{\mathcal{F}[u_0](\omega) (1 - \mathscr{L}[\psi](\eta))}{\eta (1 - \mathscr{L}[\psi](\eta) \mathcal{F}[\lambda](\omega)}.
        \end{aligned}
    \end{equation}
\end{proof}
