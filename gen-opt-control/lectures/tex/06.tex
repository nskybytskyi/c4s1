\subsection{Рівняння субдифузії}

\subsubsection{Виведення рівняння}

Розглянемо випадкове блукання із щільностями
\begin{equation}
    \psi(t) \sim \frac{\alpha}{\Gamma(1 - \alpha)} \frac{\tau^\alpha}{t^{1 + \alpha}},
\end{equation}
при $t \to + \infty$, де $\alpha \in (0, 1)$, $\tau > 0$, та
\begin{equation}
    \lambda(x) = \frac{1}{2 \sqrt{\pi} \sigma} \exp\left\{-\frac{x^2}{4 \sigma^2}\right\}.
\end{equation}

Згадуємо наслідок з теореми Таубера для $A = \frac{\tau^\alpha}{\Gamma(1 - \alpha)}$:
\begin{equation}
    \Ltrans{\psi}(\eta) \sim 1 - \tau^\alpha \eta^\alpha + o(\eta^\alpha)
\end{equation}

Крім того
\begin{equation}
    \Ftrans{\lambda}(\omega) \sim 1 - \sigma^2 \omega^2 + O(\omega^4).
\end{equation}

Застосовуємо формулу Монтрола-Вайса
\begin{equation}
    \FLtrans{u}(\omega, \eta) = \frac{\Ftrans{u_0}(\omega)}{\eta} \frac{\tau^\alpha \eta^\alpha + o(\eta^\alpha)}{1 - (1 - \tau^\alpha \eta^\alpha + o(\eta^\alpha)) (1 - \sigma^2 \omega^2 + O(\omega^4))} \sim
\end{equation}

якщо $\eta \to 0$ і $\omega \to 0$ у певному розумінні ``синхронно'', то $\eta^\alpha \omega^2 = o(\eta^\alpha + \omega^2)$, а тому
\begin{equation}
    \sim \frac{\Ftrans{u_0}(\omega)}{\eta} \frac{\tau^\alpha \eta^\alpha}{\tau^\alpha \eta^\alpha + \sigma^2 \omega^2} = \frac{\Ftrans{u_0}(\omega)}{\eta} \frac{1}{1 + \frac{\sigma^2}{\tau^\alpha} \eta^{-\alpha} \omega^2 = \frac{\Ftrans{u_0}(\omega)}{\eta} \frac{1}{1 + K_\alpha \eta^{-\alpha} \omega^2}}.
\end{equation}

Отже, з точністю до малих доданків,
\begin{equation}
    \FLtrans{u}(\omega, \eta) \cdot (1 + K_\alpha \eta^{-\alpha} \omega^2) = \frac{\Ftrans{u_0}(\omega)}{\eta},
\end{equation}
або ж,
\begin{equation}
    \FLtrans{u}(\omega, \eta) - \frac{\Ftrans{u_0}(\omega)}{\eta} = - K_\alpha \eta^{-\alpha} \omega^2 \FLtrans{u}(\omega, \eta).
\end{equation}

Оскільки
\begin{equation}
    \Ftrans{g'}(\omega) = (-i \omega) \Ftrans{g}(\omega),
\end{equation}
і, відповідно,
\begin{equation}
    \Ftrans{g^{(k)}}(\omega) = (-i \omega)^k \Ftrans{g}(\omega),
\end{equation}
то
\begin{equation}
    - \omega^2 \FLtrans{u}(\omega, \eta) = \Ftrans{\frac{\partial^2 \Ltrans{u}(x,\eta)}{\partial x^2}}
\end{equation}

Тому маємо
\begin{equation}
    \Ltrans{u}(x,\eta) - \frac{u_0(x)}{\eta} = K_\alpha \eta^{-\alpha} \frac{\partial^2 \Ltrans{u}(x,\eta)}{\partial x^2},
\end{equation}
звідки
\begin{th_equation}[субдифузії, інтегральне]
    \nothing
    \begin{equation}
        u(x, t) - u_0(x) = K_\alpha \Fint{\alpha} \left( \frac{\partial^2 u(x t)}{\partial x^2} \right),
    \end{equation}
\end{th_equation}
або
\begin{th_equation}[субдифузії, диференціальне, перша форма]
    \nothing
    \begin{equation}
        \frac{\partial u}{\partial t} = K_\alpha \RLFdiff{1 - \alpha} \left(\frac{\partial^2 u}{\partial x^2}\right),
    \end{equation}
    з початковою умовою $u(x, 0) = u_0(x)$, 
\end{th_equation}
або ж
\begin{th_equation}[субдифузії, диференціальне, друга форма]
    \nothing
    \begin{equation}
        \CFdiff{\alpha} u = K_\alpha \cdot \frac{\partial^2 u}{\partial x^2},
    \end{equation}
    з початковою умовою $u(x, 0) = u_0(x)$.
\end{th_equation}

\begin{remark}
    Випадкове блукання sз неперервним часом із
    \begin{equation}
        \psi(t) = \frac{1}{\tau} e^{-t/\tau}
    \end{equation}
    (показниковий розподіл) і з
    \begin{equation}
        \lambda \sim N(0, 2 \sigma^2)
    \end{equation}
    призвело б до класичного параболічного рівняння.
\end{remark}

\subsubsection{Аналіз рівняння}

\begin{definition}
    $\mathsf{E}[(x(t) - x(0))^2] = \langle (x(t) - x(0))^2 \rangle$ --- \textit{середньо-квадратичне зміщення} (якщо $x(0) = 0$, то $\langle x^2(t) \rangle$).
\end{definition}

\begin{lemma}
    Якщо 
    \begin{equation}
        \psi(t) \sim \frac{\alpha}{\Gamma(1 - \alpha)} \frac{\tau^\alpha}{t^{1 + \alpha}},
    \end{equation}
    при $t \to + \infty$ то
    \begin{equation}
        \langle n(t) \rangle \sim \frac{t^\alpha}{\tau^\alpha \cdot \Gamma(1 + \alpha)}.
    \end{equation}
\end{lemma}

\begin{proof}
    \begin{equation}
        \langle n(t) \rangle = \sum_{k = 0}^\infty k \mathsf{P}\{n (t) = k\} = \sum_{k = 0}^\infty k \left( \int_0^t ( \psi^{\star k}(s) - \psi^{\star (k + 1)}(s) ) \diff s \right).
    \end{equation}

    \begin{equation}
        \begin{aligned}
            \Ltrans{\langle n \rangle} (\eta)
            &= \sum_{k = 0}^\infty \frac{k}{\eta} ( (\Ltrans{\psi}(\eta))^k - (\Ltrans{\psi}(\eta))^{k + 1} ) = \\
            &= \frac{1 - \Ltrans{\psi}(\eta)}{\eta} \sum_{k = 0}^\infty k (\Ltrans{\psi}(\eta))^k = \\
            &= \frac{1 - \Ltrans{\psi}(\eta)}{\eta} \sum_{k = 0}^\infty \frac{\Ltrans{\psi}(\eta)}{\Ltrans{\psi}'(\eta)} \frac{\diff}{\diff \eta} (\Ltrans{\psi}(\eta))^k = \\
            &= \frac{1 - \Ltrans{\psi}(\eta)}{\eta} \frac{\Ltrans{\psi}(\eta)}{\Ltrans{\psi}'(\eta)} \frac{\diff}{\diff \eta} \sum_{k = 0}^\infty (\Ltrans{\psi}(\eta))^k = \\
            &= \frac{\Ltrans{\psi}(\eta)}{\eta \cdot (1 - \Ltrans{\psi}(\eta))}.
        \end{aligned}
    \end{equation}

    Оскільки
    \begin{equation}
        \Ltrans{\psi}(\eta) = 1 - \tau^\alpha \eta^\alpha + o(\eta^\alpha),
    \end{equation}
    то
    \begin{equation}
        \Ltrans{\langle n \rangle} (\eta) = \frac{1 - \tau^\alpha \eta^\alpha + o(\eta^\alpha)}{\eta \cdot (\tau^\alpha \eta^\alpha + o(\eta^\alpha))} \underset{n \to 0}{\sim} \frac{1}{\tau^\alpha \eta^{\alpha + 1}}.
    \end{equation}

    Застосовуємо ``зворотню'' теорему Таубера для $\beta = - \alpha$ отримуємо
    \begin{equation}
        \langle n(t) \rangle \sim \frac{t^\alpha}{\tau^\alpha \cdot \Gamma(1 + \alpha)}.
    \end{equation}
\end{proof}

\begin{corollary}
    Якщо $x(0) = 0$, то
    \begin{equation}
        \langle x(t)^2 \rangle = 2 \sigma^2 \langle n(t) \rangle \sim \frac{2 \sigma^2 t^\alpha}{\tau^\alpha \Gamma(1 + \alpha)} = \frac{2 K_\alpha t^\alpha}{\Gamma(1 + \alpha)}.
    \end{equation}
\end{corollary}

\begin{definition}
    Якщо у випадкового (дифузійного) процесу середньоквадратичне зміщення зростає повільніше ніж лінійна функція ($\langle x(t)^2 \rangle = o(t)$), то процес називається \textit{суб-дифузійним}.
\end{definition}

\begin{definition}
    Якщо у випадкового (дифузійного) процесу середньоквадратичне зміщення зростає швидше ніж лінійна функція ($t = o(\langle x(t)^2 \rangle)$), то процес називається \textit{супер-дифузійним}.
\end{definition}

\begin{definition}
    Якщо ж у випадкового (дифузійного) процесу середньоквадратичне зміщення зростає так само як лінійна функція, то процес називається \textit{нормально-дифузійним}.
\end{definition}

\begin{remark}
    $\langle x^2(t) \rangle$ --- середньоквадратичне зміщення, усереднене за сукупністю частинок (eng. \textit{ensemble-averaged}).
\end{remark}

Інший підхід --- усереднення за часом:
\begin{equation}
    \overline{\delta^2(t, T)} = \frac{1}{T - t} \int_0^{T - t} (x(s + t) - x(s))^2 \diff s,
\end{equation}
де $t$ --- часове вікно, $T$ --- загальна тривалість спостереження.

\begin{equation}
    \ol{\delta^2(t, T)}
    = \frac{1}{T - t} \int_0^{T - t} ( x(t + s) - x(s) )^2 \diff s
\end{equation}
--- усереднене зміщення (за часом). Усереднимо за сукупністю частинок:
\begin{equation}
    \begin{aligned}
        \ol{\langle \delta^2(t, T) \rangle}
        &= \frac{1}{T - t} \int_0^{T - t} \langle ( x(t + s) - x(s) )^2 \rangle \diff s = \\
        &= \frac{1}{T - t} \int_0^{T - t} 2 \sigma^2 \langle ( n(t + s) - n(s) )^2 \rangle \diff s.
    \end{aligned}
\end{equation}

За умов, що $t \to \infty$, $T \to \infty$ і $T \gg t$ маємо
\begin{equation}
    \langle n(t) \rangle
    \sim \frac{1}{\tau^\alpha \cdot \Gamma(1 + \alpha)} t^\alpha,
\end{equation}
а тому попередній вираз асимптотично рівний
\begin{equation}
    \frac{2 \sigma^2}{T - t} \int_0^{T - t} \frac{1}{\tau^\alpha \cdot \Gamma(1 + \alpha)} \cdot ( (s + t)^\alpha - s^\alpha ) \diff s.
\end{equation}

У свою чергу, можемо переписати $(s + t)^\alpha - s^\alpha$ за рядом Тейлора:
\begin{equation}
    (s + t)^\alpha - s^\alpha
    = s^\alpha \left( 1 + \frac{t}{s} \right)^\alpha - s^\alpha
    \sim s^\alpha \left( 1 + \frac{\alpha t}{s} \right) - s^\alpha
    = \frac{\alpha t}{s^{1 - \alpha}},
\end{equation}
а тому попередній вираз асимптотично рівний
\begin{equation}
    \begin{aligned}
        \frac{2 \sigma^2}{T - t} \cdot \frac{\alpha t}{\Gamma(1 + \alpha) \cdot \tau^\alpha} \int_0^{T - t} s^{\alpha - 1} \diff s
        &= \frac{2 \sigma^2 \alpha t}{(T - t) \cdot \Gamma(1 + \alpha) \cdot \tau^\alpha} \cdot \frac{1}{\alpha} \cdot (T - t)^\alpha = \\
        &= \frac{2 \sigma^2 t}{\Gamma(1 + \alpha) \tau^\alpha} \cdot (T - t)^{\alpha - 1} \sim \frac{2 K_\alpha}{\Gamma(1 + \alpha) \cdot T^{1 - \alpha}} t,
    \end{aligned}
\end{equation}
тобто отримали лінійну функцію від $t$. \medskip

\begin{definition}
    Ситуація, у якій $\langle x^2(t) \rangle$ і $\ol{\langle \delta^2(t, T) \rangle}$ мають різний вигляд як функції змінної $t$ називається \textit{слабкою неергодичністю} (\textit{eng.} weak ergodicity breaking).
\end{definition}

\begin{remark}
    В обмеженій області
    \begin{align}
        \langle x^2(t) \rangle &= c_1, \quad t \to \infty, \\
        \langle \delta^2(t, T) \rangle &= c_2 t^{1 - \alpha}, \quad t \to \infty,
    \end{align}
    де $c_1$, $c_2$ --- певні (можливо різні) константи.
\end{remark}

% Розглянемо тепер ситуацію коли $0 < \alpha < 1$ і $\psi(t) \sim A \alpha t^{-1 - \alpha}$, тоді математичне сподівання $\int_0^\infty t \psi(t) \diff t = \infty$ --- так званий \textit{розподіл з важким хвостом} (\textit{eng.} fat-tailed distribution).

% \begin{example}
%     Нехай $\psi(t) = \frac{t}{\tau} e^{-t / \tau}$, тоді $\langle T \rangle = \tau$.
% \end{example}
