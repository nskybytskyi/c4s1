\subsection{Початкові значення}

\subsubsection{Початкові значення інтегралів}

Дослідимо, за яких умов початкове значення інтегралу %($(\Fint{\alpha} f)(t) = o(1)$) 
дорівнює нулю.

\begin{theorem}
    Нехай $\alpha > 0$, $p > 1 / \alpha$, $p \ge 1$, $f \in L_p((0, T))$. Тоді $(\Fint{\alpha} f) (t) = o(t^{\alpha - 1 / p})$ при $t \to 0$.
\end{theorem}

\begin{proof}
    \begin{equation}
        \begin{aligned}
            | (\Fint{\alpha} f) (t) | &= \frac{1}{\Gamma(\alpha)} \left| \int_0^t f(s) (t - s)^{\alpha - 1} \diff s \right| \le \\
            &\le \frac{1}{\Gamma(\alpha)}  \int_0^t | f(s) (t - s)^{\alpha - 1} \diff s | \le \\
            &\le \frac{1}{\Gamma(\alpha)} \left( \int_0^t |f(s)|^p \diff s \right)^{1/p} \left( \int_0^t (t - s)^{(\alpha - 1) q} \diff s \right)^{1/q} = \\
            &= \frac{1}{\Gamma(\alpha)} \left( \int_0^t |f(s)|^p \diff s \right)^{1/p} \left( \frac{t^{(\alpha - 1) q + 1}}{(\alpha - 1) q + 1} \right)^{1 / q} = \\
            &= \left( \int_0^t |f(s)|^p \diff s \right)^{1/p} \frac{t^{\alpha - 1 + 1 / q}}{c(\alpha, p)} = \\
            &= \left( \int_0^t |f(s)|^p \diff s \right)^{1/p} \frac{t^{\alpha - 1 / p}}{c(\alpha, p)} = \\
            &= o(t^{\alpha - 1 / p}),
        \end{aligned}
    \end{equation}
    де останній перехід справджується адже $\int_0^t |f(s)|^p \diff s = o(1)$ при $t \to 0$.
\end{proof}

\begin{reminder}
    \nothing
    \begin{proposition}[абсолютна неперервність інтеграла Лебега]
        Якщо $f \in L_1$ то
        \begin{equation}
            (\forall \epsilon > 0) \quad (\exists \delta(\epsilon) > 0) \quad (\forall A: \mu(A) < \delta(\epsilon)) \quad \int_A f \diff \mu \le \epsilon.
        \end{equation}
    \end{proposition}

    \begin{th_inequality}[Коші-Буняковського, інтегральна]
        Якщо всі функції достатньо інтегровні (всі норми скінченні)
        \begin{equation}
            \|f \cdot g\|_{L_1} \le \|f\|_{L_2} \cdot \|g\|_{L_2}.  
        \end{equation}
    \end{th_inequality}
    
    \begin{th_inequality}[Гельдера, інтегральна]
        Якщо всі функції достатньо інтегровні (всі норми скінченні)
        \begin{equation}
            \|f \cdot g\|_{L_1} \le \|f\|_{L_p} \cdot \|g\|_{L_q},  
        \end{equation}
        де $1 / p + 1 / q = 1$.
    \end{th_inequality}
\end{reminder}

\begin{remark}
    Умова $p > 1 / \alpha$ необхідна для збіжності усіх інтегралів з доведення
\end{remark}

\begin{corollary}
    При $\alpha > 1 / p$ маємо $(\Fint{\alpha} f)(t) = o(1)$, тобто $(\Fint{\alpha} f)(0) = 0$.
\end{corollary}

\begin{exercise}
    Наведіть приклад $f$ для якої $(\Fint{\alpha} f)(0) \ne 0$ (але і не $\infty$).
\end{exercise}
% \begin{solution}
%     \begin{example}
%         Розглянемо $f(x) = \frac{1}{\sqrt{x}}$ на $[0,1]$. Можна показати, що
%         \begin{equation}
%             \int_0^1 |f(x)| \diff x = 2,
%         \end{equation}
%         тому $f \in L_1((0,1))$. З іншого боку,
%         \begin{equation}
%             \int_0^1 |f(x)|^2 \diff x = \int_0^1 \frac{\diff x}{x} = \infty,
%         \end{equation}
%         тому $f \not\in L_2((0,1))$. Це означає, що $1 < p < 2$. Тоді нерівність $p > 1 / \alpha$ з умов теореми не буде виконуватися, для $\alpha \le \frac{1}{2}$. Зокрема, виникають певні сподівання на $\alpha = \frac{1}{2}$. Розглянемо $(\Fint{1/2}f)(t)$:
%         \begin{equation}
%             (\Fint{1/2}f)(t) = \frac{1}{\Gamma(1/2)} \int_0^t \frac{\diff s}{\sqrt{t - s} \sqrt{s}} = \frac{\pi}{\Gamma(1/2)} = \sqrt{\pi}.
%         \end{equation}
%         Як бачимо, отриманий вираз не просто не прямує до нуля при $t \to 0$, а взагалі не залежить від $t$, тобто наші сподівання не були марні і $(\Fint{1/2}f)(0) = \sqrt{\pi} \ne 0$ (але і не $\infty$).
%     \end{example}
% \end{solution}

\subsubsection{Початкові значення похідних}

\begin{theorem}
    Нехай $\alpha > 0$, $\alpha \not\in \NN$, $n = \lceil \alpha \rceil$, $f \in C^{n - 1}([0, T])$, $p > \frac{1}{n - \alpha}$, $f^{(n)} \in L_p([0, T])$. Тоді $(\RLFdiff{\alpha}) (0) = 0 \iff f^{(k)} (0) = 0$ при $k = \overline{0, n - 1}$.
\end{theorem}

\begin{proof}
    За умов теореми
    \begin{equation}
        \label{eq:3-to-ref}
        (\RLFdiff{\alpha} f)(t) = \frac{1}{\Gamma(n - \alpha)} \int_0^t \frac{f^{(n)}(s)}{(t - s)^{\alpha - n + 1}} \diff s + \sum_{k = 0}^{n - 1} \frac{f^{(k)}(0) \cdot t^{k - \alpha}}{\Gamma(k - \alpha + 1)}.
    \end{equation}

    ($\Longleftarrow$) У формулі вище інтеграл дорівнює нулю за першою сьогоднішньою теоремою, а уся сумма зануляється за умовою теореми. \medskip

    ($\Longrightarrow$) Домножатимемо \eqref{eq:3-to-ref} на $t^{\alpha - k}$ для $k = \overline{0, n - 1}$. Наприклад, для $k = 0$ матимемо
    \begin{equation}
        t^\alpha (\RLFdiff{\alpha} f)(t) = t^\alpha (\CFdiff{\alpha} f)(t) + \frac{f(0)}{\Gamma(1 - \alpha)} + \sum_{k = 1}^{n - 1} \frac{f^{(k)}(0) \cdot t^k}{\Gamma(k - \alpha + 1)}.
    \end{equation}

    Бачимо, що $t^\alpha (\CFdiff{\alpha} f)(t) = o(1)$, всі доданки суми нескінченно малі, тому $f(0) = 0$. Далі за індукцією по $k$ отримуємо рівність нулеві усіх похідних до $(n - 1)$-ої.
\end{proof}

\begin{remark}
    При $0 < \alpha < 1$ маємо $(\RLFdiff{\alpha} \textbf{1})(t) = \frac{1}{\Gamma(1 - \alpha) t^\alpha} \ne 0$.
\end{remark}

\begin{remark}
    Але $(\CFdiff{\alpha} \textbf{1})(t) = 0$.
\end{remark}

\begin{theorem}
    Нехай $\alpha > 0$, $n = \lceil \alpha \rceil$, $f \in C^n([0, T])$, тоді
    \begin{equation}
        \RLFdiff{\alpha} f \equiv 0 \iff f(t) = \sum_{k = 0}^{n - 1} c_k t^{\alpha - k - 1}
    \end{equation}
    --- дробовий многочлен.
\end{theorem}
\begin{proof}
    \nothing
    \begin{exercise}
        ($\Longrightarrow$)
        % За умов теореми 
        % \begin{equation}
        %     0 = (\Fint{\alpha} 0)(t) = (\Fint{\alpha} \RLFdiff{\alpha} f)(t) = f(t) - \sum_{k = 0}^{n - 1} (\RLFdiff{\alpha - k  - 1} f)(0) \cdot \frac{t^{\alpha - k - 1}}{\Gamma(\alpha - k)},
        % \end{equation}
        % звідки
        % \begin{equation}
        %     f(t) = \sum_{k = 0}^{n - 1} (\RLFdiff{\alpha - k - 1} f)(0) \cdot \frac{t^{\alpha - k - 1}}{\Gamma(\alpha - k)},
        % \end{equation}
        % і, перепозначаючи $c_k = \frac{\RLFdiff{\alpha - k - 1} f)(0)}{\Gamma(\alpha - k)}$, отримуємо якраз ту форму для $f$, яку хотіли.
    \end{exercise}
    
    ($\Longleftarrow$) Нехай
    \begin{equation}
        f(t) = \sum_{k = 0}^{n - 1} c_k t^{\alpha - k - 1},
    \end{equation}
    тоді 
    \begin{equation}
        \RLFdiff{\alpha} f = \frac{\diff^n}{\diff t^n} \Fint{n - \alpha} \left( \sum_{k = 0}^{n - 1} t^{\alpha - k - 1} \right) = \frac{\diff^n}{\diff t^n} \sum_{k = 0}^{n - 1} \frac{\Gamma(\alpha - k)}{\Gamma(n - k)} \cdot t^{n - k - 1} = 0.
    \end{equation}
\end{proof}

\begin{theorem}[похідна добутку]
    Нехай $f$, $g$ --- аналітичні в $(-h, h)$. Тоді для $t \in (0, h/2)$
    \begin{equation}
        ( \RLFdiff{\alpha} (f \cdot g) ) (t) = \sum_{k = 0}^\infty \binom{k}{\alpha} ( \RLFdiff{k} f ) (t) ( \RLFdiff{\alpha - k} f ) (t),
    \end{equation}
    де
    \begin{equation}
        \binom{k}{\alpha} = \frac{\Gamma(\alpha + 1)}{\Gamma(k + 1) \cdot \Gamma(\alpha - k - 1)}.
    \end{equation}
\end{theorem}

\begin{theorem}[Тарасова]
    Нехай $0 < \alpha < 1$, тоді $\RLFdiff{\alpha}$ --- лінійний оператор, що задовольняє умову
    \begin{equation}
        \RLFdiff{\alpha}(f \cdot g) = \RLFdiff{\alpha} f \cdot g + f \cdot \RLFdiff{\alpha} g.
    \end{equation}
    
    Тоді $\exists p(t)$: $(\RLFdiff{\alpha} f) (t) = p(t) \cdot \frac{\diff f}{\diff t}$.
\end{theorem}
