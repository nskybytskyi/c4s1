\subsection{Рівняння реакції-субдифузії змінного порядку}

Згадаємо класичне рівняння реакції-дифузії:
\begin{equation}
    \frac{\partial u}{\partial t} \equiv k \Delta u - \theta u,
\end{equation}
де $\theta$ --- кооефіцієнт реакції (реакція це процес у якому частинки речовини зникають). \medskip

Наївне узагальнення:
\begin{equation}
    \CFdiff{\alpha} u \equiv k_\alpha \Delta u - \theta u.
\end{equation}

\begin{remark}
    Основна проблема із цим рівнянням у тому, що його розв'язок $u(x, t)$, взагалі кажучи, не є невід'ємним, навіть якщо $u_0(x) \ge 0$.
\end{remark}

Скористаємося напів-дискретним підходом: розглянемо сітку з рівновіддаленими вузлами (для наочності --- у одновимірному випадку). Нехай $u_i(t)$ --- кількість частинок речовини у $i$-ому вузлі у момент часу $t$. Будемо вважати, що стрибки відбуваються в один із сусідніх вузлів із ймовірностями $1/2$ (тобто блукання не зміщене). Нехай також, як і раніше, $\psi_i(t)$ --- щільність часу очікування стрибка у вузлі $i$. \medskip

Також вважаємо, що час зникнення частинки має показниковий розподіл з параметром $\theta_i$. Показниковий розподіл особливий тим, що у нього відсутній ефект післядії: ймовірність розпаду у проміжку часу $[t_0, t_0 + \Delta t]$ не залежить від $t_0$ і дорівнює $1 - e^{-\theta_i t}$, а ймовірність продовження існування дорівнює $e^{-\theta_i t}$. \medskip

Це рівносильно тому, що за відсутності стриків (тобто без дифузії) рівняння мало б такий вигляд:
\begin{equation}
    \frac{\diff u_i}{\diff t} \equiv -\theta_i u_i,
\end{equation}
адже розв'язок цього рівняння має вигляд
\begin{equation}
    u_i(t) = u_i(0) \cdot e^{-\theta_i t},
\end{equation}
тобто отримали (з точністю до множника) ймовірність продовження існування для показникового розподілу. \medskip

Введемо ще дві величини: 
\begin{definition}
    \textit{Вхідний та вихідний інтегральні потоки} $J_i^+(t)$, $J_i^-(t)$ такі, що
    \begin{equation}
        \int_{t_1}^{t_2} J_i^+(t) \diff t
    \end{equation}
    --- кількість частинок, що прибули в $i$-ий вузол впродовж часу $[t_1, t_2]$, а
    \begin{equation}
        \int_{t_1}^{t_2} J_i^-(t) \diff t
    \end{equation}
    --- кількість частинок, що вибули з $i$-ого вузла за час $[t_1, t_2]$.
\end{definition}

Відносно них ми і запишемо рівняння: за наведених припущень, маємо такі рівняння:
\begin{equation}
    \frac{\diff u_i}{\diff t} \equiv J_i^+ - J_i^- - \theta_i u_i
\end{equation}
а також
\begin{equation}
    J_i^+ \equiv \tfrac{1}{2} J_{i - 1}^- + \tfrac{1}{2} J_{i + 1}^-.
\end{equation}

Безпосередньо з цих двох рівнянь випливає, що
\begin{equation}
    \frac{\diff u_i}{\diff t} \equiv \tfrac{1}{2} J_{i - 1}^- - J_i^- + \tfrac{1}{2} J_{i + 1}^- - \theta_i u_i.
\end{equation}

Крім того,
\begin{equation}
    J_i^-(t) = u_i(0) \cdot e^{-\theta_i t} \psi_i(t) + \int_0^t J_i^+(s) \cdot e^{-\theta_i (t - s)} \cdot \psi_i(t - s) \diff s.
\end{equation}

Перший доданок відповідає за частинки, які з самого початку були в $i$-ому вузлі, не розпалися за час $t$ (ймовірність цього $e^{-\theta_i t}$) і вистринули з нього у час $t$ (ймовірність цього $\psi_i(t)$), а підінтегральний вираз у другому --- за  ті частинки, які прибули у момент часу $s$, не розпалися за час $t - s$ (ймовірність цього $e^{-\theta_i (t - s)}$), і вистрибнули з нього через час $t - s$ після прибуття (ймовірність цього $\psi_i(t - s)$). \medskip

Перетворимо останнє рівняння перетворенням Лапласа:
\begin{equation}
    \begin{aligned}
        \Ltrans{J_i^-}(\eta)
        &= u_i(0) \cdot \Ltrans{\psi_i}(\eta + \theta_i) + \Ltrans{J_i^+}(\eta) \cdot \Ltrans{\psi_i}(\eta + \theta_i) = \\
        &= u_i(0) \cdot \Ltrans{\psi_i}(\eta + \theta_i) + ( \eta \cdot \Ltrans{u_i}(\eta) - u_i(0) + \\
        &\quad + \Ltrans{J_i^-}(\eta) + \theta_i \cdot \Ltrans{u_i}(\eta) ) \cdot \Ltrans{\psi_i}(\eta + \theta_i) = \\
        &= ( \eta \cdot \Ltrans{u_i}(\eta) + \Ltrans{J_i^-}(\eta) + \theta_i \cdot \Ltrans{u_i}(\eta) ) \cdot \Ltrans{\psi_i}(\eta + \theta_i).
    \end{aligned}
\end{equation}

Звідси
\begin{equation}
    \begin{aligned}
        \Ltrans{J_i^-}(\eta)
        &= \frac{(\eta + \theta_i) \cdot \Ltrans{u_i}(\eta) \cdot \Ltrans{\psi_i}(\eta + \theta_i)}{1 - \Ltrans{\psi_i}(\eta + \theta_i)} = \\
        &= \eta \cdot \Ltrans{u_i}(\eta) \cdot \Ltrans{e^{-\theta_i t} \cdot \iLtrans{\frac{\Ltrans{\psi_i}(\eta)}{1 - \Ltrans{\psi_i}(\eta)}}} + \\
        &\quad + \theta_i \cdot \Ltrans{u_i}(\eta) \cdot \Ltrans{e^{-\theta_i t} \cdot \iLtrans{\frac{\Ltrans{\psi_i}(\eta)}{1 - \Ltrans{\psi_i}(\eta)}}}.
    \end{aligned}
\end{equation}

Нехай тепер $\psi_i(t) \sim r_i \cdot t^{-1 - \alpha_i}$, де $0 < \alpha_i < 1$. Тоді $\Ltrans{\psi_i}(\eta) \sim 1 - r_i \cdot \frac{\Gamma(1 - \alpha_i)}{\alpha_i} \cdot \eta^{\alpha_i} + o(\eta^{\alpha_i})$ (за наслідком з теореми Таубера). Отже,
\begin{equation}
    \frac{\Ltrans{\psi_i}(\eta)}{1 - \Ltrans{\psi_i}(\eta)} = \frac{1 - r_i \cdot \frac{\Gamma(1 - \alpha_i)}{\alpha_i} \cdot \eta^{\alpha_i} + o(\eta^{\alpha_i})}{r_i \cdot \frac{\Gamma(1 - \alpha_i)}{\alpha_i} \cdot \eta^{\alpha_i} + o(\eta^{\alpha_i})} \sim \frac{\alpha_i \cdot \eta^{-\alpha_i}}{r_i \cdot \Gamma(1 - \alpha_i)} = M_i \cdot \eta^{-\alpha_i}.
\end{equation}

За теоремою Таубера з $\beta = 1 - \alpha_i$,
\begin{equation}
    \iLtrans{\frac{\Ltrans{\psi_i}(\eta)}{1 - \Ltrans{\psi_i}(\eta)}} \sim \frac{M_i}{\Gamma(\alpha_i)} \cdot t^{\alpha_i - 1}.
\end{equation}

Тому можемо продовжити
\begin{equation}
    J_i^-(t) \sim \frac{\diff}{\diff t} \left( u_i(t) \star e^{-\theta_i t} \cdot \frac{M_i}{\Gamma(\alpha_i)} \cdot t^{\alpha_i - 1} \right) + \theta_i \cdot \left( u_i(t) \star e^{-\theta_i t} \cdot \frac{M_i}{\Gamma(\alpha_i)} \cdot t^{\alpha_i - 1} \right).
\end{equation}

\begin{remark}
    У цій формулі не фігурує початкове значення, адже воно рівне нулеві для достатньо гладкої $u_i$, наприклад для обмеженої в околі нуля і інтегровної. Справді, тоді $\left. u_i \star e^{-\theta_i t} t^{\alpha_i - 1} \right|_{t = 0}$:
    \begin{equation}
        \left. u_i \star e^{-\theta_i t} \cdot t^{\alpha_i - 1} \right|_{t = 0} = \int_0^t u_i(s) e^{-\theta_i (t - s)} \cdot (t - s)^{\alpha_i - 1} \diff s \le \int_0^t \|u\| \cdot 1 \cdot (t - s)^{\alpha_i - 1} \diff s \xrightarrow[t \to 0]{} 0.
    \end{equation}
\end{remark}

Ми майже досягнули нашої мети:
\begin{equation}
    \begin{aligned}
        \frac{1}{M_i} \cdot J_i^-(t)
        &= \frac{\diff}{\diff t} \frac{1}{\Gamma(\alpha_i)} \int_0^t u_i(s) \cdot e^{-\theta_i (t - s)} \cdot (t - s)^{\alpha_i - 1} \diff s + \\
        &\quad + \theta_i \cdot \frac{1}{\Gamma(\alpha_i)} \int_0^t u_i(s) \cdot e^{-\theta_i (t - s)} \cdot (t - s)^{\alpha_i - 1} \diff s = \\
        &= \frac{\diff}{\diff t} e^{-\theta_i t} \cdot \frac{1}{\Gamma(\alpha_i)} \int_0^t u_i(s) e^{\theta_i s} \cdot (t - s)^{\alpha_i - 1} \diff s + \\
        &\quad + \theta_i \cdot \frac{1}{\Gamma(\alpha_i)} \int_0^t u_i(s) \cdot e^{-\theta_i (t - s)} \cdot (t - s)^{\alpha_i - 1} \diff s = \\
        &= \frac{\diff}{\diff t} e^{-\theta_i t} \cdot \frac{1}{\Gamma(\alpha_i)} \int_0^t u_i(s) \cdot e^{\theta_i s} \cdot (t - s)^{\alpha_i - 1} \diff s + \\
        &\quad + \theta_i \cdot \frac{1}{\Gamma(\alpha_i)} \int_0^t u_i(s) \cdot e^{-\theta_i (t - s)} \cdot (t - s)^{\alpha_i - 1} \diff s = \\
        &= e^{-\theta_i t} \cdot \frac{\diff}{\diff t} \frac{1}{\Gamma(\alpha_i)} \int_0^t u_i(s) \cdot e^{\theta_i s} \cdot (t - s)^{\alpha_i - 1} \diff s = \\
        &= e^{-\theta_i t} \cdot \RLFdiff{1 - \alpha_i} (e^{\theta_i t} \cdot u_i(t)).
    \end{aligned}
\end{equation}

Пісдтавляємо це назад:
\begin{equation}
    \frac{\diff u_i}{\diff t} = \frac{M_{i - 1}}{2}.    
\end{equation}

\begin{equation}
    \frac{1}{M_i} \cdot J_i^- = e^{-\theta_i t} \cdot \RLFdiff{1 - \alpha_i}(e^{\theta_t t} u_i(t)),
\end{equation}
де
\begin{equation}
    M_i = \frac{\alpha_i}{r_i - \Gamma(1 - \alpha_i)},
\end{equation}
а $\alpha_i$ береться з 
\begin{equation}
    \psi_i(t) \sim r_i \cdot t^{1 - \alpha_i}.
\end{equation}


Звідси маємо 
\begin{th_equation}[реакції-субдифузії, напівдискретне]
    \nothing
    \begin{equation}
        \begin{aligned}
            \frac{\diff u_i(t)}{\diff t}
            &= \tfrac{1}{2} M_{i - 1} \cdot e^{-\theta_{i - 1} t} \cdot \RLFdiff{1 - \alpha_{i - 1}} (e^{\theta_{i - 1} t} u_{i - 1}) + \\
            &\quad + \tfrac{1}{2} M_{i + 1} \cdot e^{-\theta_{i + 1} t} \cdot \RLFdiff{1 - \alpha_{i + 1}} (e^{\theta_{i + 1} t} u_{i + 1}) + \\
            &\quad - M_i \cdot e^{-\theta_i t} \cdot \RLFdiff{1 - \alpha_i}(e^{\theta_i t} u_i) - \theta_i u_i.
        \end{aligned}
    \end{equation}
\end{th_equation}

Граничний перехід до неперервної за простором моделі:
\begin{itemize}
    \item $i \in \ZZ \mapsto x \in \RR$;
    \item $u(t) \mapsto u(x, t)$;
    \item $\alpha_i \mapsto \alpha(x)$;
    \item $\theta_i \mapsto \theta(x)$;
    \item $r_i \mapsto r(x)$.
\end{itemize}

Якщо все це обережно проробити то отримаємо
\begin{th_equation}[реакції субдифузії, змінного порядку]
    \nothing
    \begin{equation}
        \label{eq:starstar}
        \frac{\partial u}{\partial t} = \frac{\partial^2}{\partial x^2} \left( k(x) \cdot e^{-\theta(x) t} \cdot \RLFdiff{1 - \alpha(x)}\left( e^{\theta(x) t} \cdot u \right) \right) - \theta(x) \cdot u.
    \end{equation}
    
    Тут 
    \begin{equation}
        k(x) = \frac{\alpha(x) \cdot \sigma^2}{2 \cdot r(x) \cdot \Gamma(1 - \alpha(x))}
    \end{equation}
    --- \textit{як-би коефіцієнт дифузії}.
\end{th_equation}

\begin{remark}
    Нагадаємо, що класичне рівняння реакції дифузії мало вигляд
    \begin{equation}
        \frac{\partial u}{\partial t} \equiv k \Delta u - \theta u.
    \end{equation}
    
    Як бачимо, результат переходу до дробових похідних анітрохи не очевидний, тобто рівняння потрібно виводити, а не вгадувати.
\end{remark}

Якщо $\theta = 0$ (реакції немає), то маємо рівняння субдифузії змінного порядку:
\begin{equation}
    \label{eq:star}
    \frac{\partial u}{\partial u} = \frac{\partial^2}{\partial x^2} ( k(x) \RLFdiff{1 - \alpha(x)}(u) ).
\end{equation}

\begin{remark}
    Причому $\RLFdiff{1 - \alpha(x)}$ не можна винести за $\frac{\partial^2}{\partial x^2}$, тобто останнє рівняння не еквівалентне такому:
    \begin{equation}
        \CFdiff{\alpha(x)} u = k(x) \cdot \frac{\partial^2 u}{\partial x^2}.
    \end{equation}
    
    А от у рівняння субдифузії для достатньо гладких функцій можна було.
\end{remark}

\begin{remark}
    Якщо у моделі реакції-субдифузії $\theta = \theta(x, t, u(x, t))$, то отримаємо рівняння аналогічне \eqref{eq:starstar}, але з
    \begin{equation}
        \exp\left\{ \pm \int_0^t \theta(x, s, u(x, s)) \diff s \right\}
    \end{equation}
    замість $e^{\pm \theta(x) t}$.
\end{remark}

\begin{remark}
    Якщо $\alpha$ --- стала, то отримуємо старе рівняння субдифузії: рівняння \eqref{eq:star} з $\alpha(x) = \alpha = \const$ з $r(x) = \frac{\alpha}{\Gamma(1 - \alpha)} \cdot \tau^\alpha$ зводиться до рівняння субдифузії
    \begin{equation}
        \frac{\partial u}{\partial t} = k_\alpha \cdot \frac{\partial^2}{\partial x^2} \RLFdiff{1 - \alpha} u     
    \end{equation} 
    з $k_\alpha = \frac{\sigma^2}{2 \tau^\alpha}$.
\end{remark}
