\section{Точні та узагальнені розв'язки диференціальних та різницевих рівнянь у банахових просторах}

\shortLectureDescription{Поняття точного та узагальненого розв'язків диференціальних та різницевих рівнянь у банахових просторах. Розв'язуючі оператори. Коректність постановок задач. Збіжність та стійкість. Терема Лакса. \cite{gryschenko2005, richtmyer1972}}

Нехай $B$ --- нормований банаховий простір. Функції, які залежать від просторових змінних $x$ і координати часу $t$, при фіксованому $t$ будемо тлумачити як точки цього функціонального простору і позначати їх символом $u$. Стан фізичної системи зображатимемо точкою функціонального простору, а її положення в часі відображатиме рух цієї точки у функціональному просторі $B$. Через $\tilde{\RR}_n$  позначимо $n$-вимірний комплексний евклідів векторний простір зі скалярним добутком
\begin{equation*}
    \left(\vec a, \vec b\right) = \sum_{j = 1}^n a_j \bar b_j
\end{equation*}
і нормою
\begin{equation*}
    \|\vec a\| = \sqrt{\sum_{j = 1}^n a_j \bar a_j}.
\end{equation*}

Введемо також допоміжний нормований простір $B'$, елементами якого є довільні неперервні криві, які при кожному $t \in [0, T]$, визначені як функції просторових змінних $w(t) \in B$. В $B'$ введемо норму так:
\begin{equation*}
    \|w\|_{B'} = \max_{0 \le t \le T} \|w(t)\|_B.
\end{equation*}

\subsection{Cкінченно-різницеві задачі в абстрактних просторах Банаха.Точні і узагальнені розв'язки диференціальних рівнянь}

Розглянемо однопараметричне сімейство елементів $u(t) \in B$ з дійсним параметром $t$ таких, що 
\begin{equation}
    \label{eq:3.1}
    \frac{\partial u(t)}{\partial t} = A u(t); \quad u(0) = u_0; \quad 0 \le t \le T,
\end{equation}
де $u_0$ --- заданий елемент з $B$, який характеризує початковий стан системи; $A$ --- лінійний оператор. \medskip

\begin{definition}
    \textit{Точний розв'язок} задачі визначимо як однопараметричне сімейство $u(t)$, кожен елемент якого належить області визначення оператора $A$, $\forall t \in [0, T]$, $u(0) = u_0$ і
    \begin{equation}
        \label{eq:3.2}
        \left\| \frac{u(t + \Delta t) - u(t)}{\Delta t} - A u(t) \right\| \to 0,
    \end{equation}
    при $\Delta t \to 0$, $\forall t \in [0, T]$.
\end{definition}

Позначимо через $D$ множину елементів $u_0 \in B$, для кожного з яких існує єдиний розв'язок задачі \eqref{eq:3.1} при $u(0) = u_0$, а збіжність в \eqref{eq:3.2} рівномірна за $t$. Нехай $E_0(T)$ відображає $D$ в $B$ і при фіксованому $t$ встановлює відповідність між $u_0$ і $u(t)$. Тоді $u(t) = E_0(T) u_0$ є розв'язком задачі для тих $u_0 \in B$, для яких існує точний розв'язок.

\begin{definition}
    Задачі, визначені лінійним оператором $A$, назвемо \textit{коректними за Адамаром}, якщо: 
    \begin{enumerate}
        \item область визначення $D$ перетворення $E_0(t)$ щільна в $B$;
        \item сімейство перетворень $E_0(t)$ рівномірно обмежене, тобто існує така додатна стала $K \in \RR$, що $\|E_0(t)\| \le K$ при $0 \le t \le T$. 
    \end{enumerate}
\end{definition}

Перша умова стверджує, що якщо для деякого початкового значення $u_0$ точний розв'язок не існує, то цей початковий елемент можна апроксимувати як завгодно точно з допомогою тих початкових елементів з $D$, для яких існує точний розв'язок. З другої умови випливає, що розв'язок задачі неперервно залежить від початкового значення. \medskip

\begin{theorem}[теорема про розширення оператора \cite{richtmyer1972}]
    Обмежений лінійний оператор $T$, область визначення якого щільна в $B$, має єдине лінійне обмежене розширення $T'$, область визначення якого співпадає з $B$, і таке, що $\|T\| = \|T'\|$.
\end{theorem}

\begin{definition}
    Обмежений лінійний оператор $E_0(t)$ з щільною в $B$ областю визначення має єдине розширення $E(t)$, яке назвемо \textit{узагальненим розв'язуючим оператором}.
\end{definition}

Оператор $E(t)$ визначений на всьому просторі $B$, обмежений за нормою тим же числом $K$, що і оператор $E_0(t)$. 

\begin{definition}
    Рівність 
    \begin{equation*}
        % \label{eq:3.3}
        u(t) = E(t) u_0
    \end{equation*}
    є \textit{узагальненим розв'язком задачі} для довільного початкового елемента $u_0 \in B$.
\end{definition}

Якщо оператор $A$ явним чином залежить від часу, то узагальнений розв'язуючий оператор стає функцією двох змінних. Дійсно, оскільки в момент часу $t_0$ задано початковий стан $u_0$, то $u(t) = E(t, t_0) u_0$ і в силу напівгрупової властивості $E(t_2, t_0) = E(t_2, t_1) E(t_1, t_0)$ при $t_0 \le t_1 \le t_2$. \medskip

Розглянемо неоднорідну крайову задачу
\begin{equation}
    \label{eq:3.3}
    \frac{\partial u(t)}{\partial t} - A u(t) = g(t), \quad u(0) = u_0,
\end{equation}
де $u_0$ і $g(t)$ задані, а $g(t)$ рівномірно або кусково рівномірно неперервна (в  нормі простору $B$) за часом $t$ функція на відрізку $0 \le t \le T$. Вважаємо, що оператор $A$, який визначає коректно поставлену однорідну задачу, замкнений, а області визначення всіх його степенів щільні в $B$. \medskip

Наступні твердження доводяться в \cite{richtmyer1972}:
\begin{proposition}
    Якщо 
    \begin{enumerate}
        \item $u_0$ і $g(t) \in D(A)$; 
        \item $g(t) \in D(A^2)$; 
        \item функції $A g(t)$ і $A^2 g(t)$ неперервні,
    \end{enumerate}
    то 
    \begin{equation}
        \label{eq:3.4}
        u(t) = E(t) u_0 + \int_0^t E(t - s) g(s) \diff s
    \end{equation}
    є \textit{точним розв'язком задачі} \eqref{eq:3.3}.
\end{proposition}

\begin{proposition}
    Якщо на $u_0$ і $g(t)$  не накладено ніяких обмежень, крім неперервності $g(t)$, або ці умови зводяться тільки до умови існування інтегралу в \eqref{eq:3.4}, то \eqref{eq:3.4} є узагальненим розв'язком задачі \eqref{eq:3.3}.
\end{proposition}

\begin{proposition}
    При вказаних вимогах відносно оператора $A$ і функції $g(t)$ узагальнений розв'язок існує.
\end{proposition}

\begin{remark}
    Єдиність розв'язку випливає з єдиності однорідної задачі, яка за припущенням поставлена коректно. 
\end{remark}

\subsection{Скінченно-різницеві задачі}

Розглянемо однорідне скінченно-різницеве рівняння 
\begin{equation}
    \label{eq:3.4'}
    B_1 u^{n + 1} = B_0 u^n,
\end{equation}
де $B_0 = B_0(\Delta t, \Delta x_1, \Delta x_2, \ldots)$ і $B_1 = B_1(\Delta t, \Delta x_1, \Delta x_2, \ldots)$ --- лінійні скінченно–різницеві оператори, залежні від приростів $\Delta t, \Delta x_1, \Delta x_2, \ldots$ і можливо від просторових змінних. Обидві частини рівняння є лінійними функціями значень $u$, визначених в точках з деякої множини (шаблону). \medskip

Нехай існує обернений оператор $B_1^{-1}$, $B_1^{-1} B_0$ обмежений і вони визначені на всьому $B$, а $\Delta x_i = q_i(\Delta t)$, $i = \overline{1,d}$, де $d$ --- розмірність простору. Позначимо $B_1^{-1}(\Delta t, \Delta x_1, \Delta x_2, \ldots) \cdot B_0(\Delta t, \Delta x_1, \Delta x_2, \ldots) = C(\Delta t)$, тоді 
\begin{equation}
    \label{eq:3.5}
    u^{n + 1} = C(\Delta t) u^n.
\end{equation}

\begin{definition}
    Сімейство операторів $C(\Delta t)$ \textit{узгоджено апроксимує} крайову задачу \eqref{eq:3.1}, якщо для довільного $u(t)$ з деякого класу $U$ точних розв'язків, початкові елементи яких утворюють в $B$ щільну множину, справедлива \textit{умова узгодження}:
    \begin{equation}
        \label{eq:3.6a}
        \left\| \left( \frac{C(\Delta t) - I)}{\Delta t} - A \right) u(t) \right\| \to 0,
    \end{equation}
    при $\Delta t \to 0$, $0 \le t \le T$.
\end{definition}

Тут $I$ --- одиничний оператор. \medskip

\begin{definition}
    Враховуючи \eqref{eq:3.2}, маємо
    \begin{equation}
        \label{eq:3.6b}
        \left\| \frac{u(t + \Delta t) - C(\Delta t) u(t)}{\Delta t} \right\| \to 0,
    \end{equation}
    при $\Delta t \to 0$, $0 \le t \le T$, в якому вираз під знаком норми є \textit{похибкою апроксимації}.
\end{definition}

Якщо $\forall \epsilon > 0$ $\exists \delta > 0$ таке, що 
\begin{equation}
    \label{eq:3.6c}
    \left\| \left( C(\Delta t) - E(\Delra t) \right) u(t) \right\| < \epsilon \Delta t,
\end{equation}
при $0 \le t \le T$, $0 < \Delta t < \delta$, то \textit{збіжність рівномірна} за часом $t$. \medskip

\begin{definition}
    Сімейство $C(\Delta t)$ забезпечує \textit{збіжну апроксимацію} задачі, якщо для будь-якого фіксованого $t \in [0, T]$, для кожного $u_0 \in B$ і для кожної збіжної до нуля послідовності додатних приростів $\{\Delta_j t\}_{j = 1}^\infty$, має місце граничне відношення 
    \begin{equation}
        \label{eq:3.7}
        \left\| C(\Delta_j t)^{n_j} u_0 - E(t) u_0 \right\| \to 0
    \end{equation}
    при $j \to \infty$, де $n_j \in \NN$ такі, що $n_j \Delta_j t \to t$ при $j \to \infty$.
\end{definition}

\begin{definition}
    Апроксимацію $C(\Delta t)$ назвемо \textit{стійкою}, якщо для деякого $\tau > 0$ нескінченна множина операторів 
    \begin{equation}
        \label{eq:3.9}
        C(\Delta t)^n, \quad 0 \le n \Delta t \le T, \quad 0 < \Delta t < \tau
    \end{equation}
    рівномірно обмежена. 
\end{definition}

\begin{theorem}[Лакса, про еквівалентність]
    Нехай задача \eqref{eq:3.1}--\eqref{eq:3.2} коректно поставлена та її скінченно-різницева апроксимація задовольняє умову узгодження. Тоді стійкість необхідна і достатня для збіжності.
\end{theorem}

\begin{proof}
    \textbf{Необхідність.} Спочатку ми покажемо, що збіжна схема необхідно є стійкою. Ми cтверджуємо, що для всякої збіжної схеми й для довільного початкового фіксованого елемента   величини
 , ( ,  ),
обмежені при деякому   . Дійсно, якщо це не так, то знайдуться дві послідовності    і   , для яких норми елементів  
необмежено зростають (при цьому   повинні прямувати до нуля в силу припущення про неперервну залежність   від додатних значень  ); із цих елементів ми можемо вибрати підпослідовність, для якої величини   збігаються до деякого    з відрізка  ;
але це суперечить припущенню про збіжність схеми, оскільки при наявності збіжності норми елементів цієї підпослідовності повинні були б прямувати до кінцевої границі  . Отже, існує така функція  , що неперервність   виконується для всіх операторів з множини (9) і всіх  , отже множина (9) рівномірно обмежена. Таким чином, апроксимація стійка. \medskip

\textbf{Достатність.} Щоб довести зворотне твердження, припустимо, що   є точним розв'язком, що належать класу  , про який йшла мова при визначенні узгодженості. Нехай   ті ж, що й в умові узгодженості у формі (6с),   і   обрані так само, як і при визначенні збіжності, а   позначає різницю між обчисленим і точним значенням і в момент часу  , тобто
            (10)
Третя частина цієї рівності, співпадає з другою: після приведення подібних залишаються тільки перший і останній члени. Норму величини   можна оцінити за допомогою (6с) і нерівності трикутника: якщо  , то
       ,    (11)
де   позначає рівномірну границю множини (9). Так як    довільно, то    при  ,  . Для доведення збіжності покажемо, що в граничному переході при   в (10) можна замінити   на  . Якщо  ,  , то в силу напівгрупової властивості сімейства   маємо   так що  , причому знак визначається знаком різниці  . У будь-якому випадку
      ,
де   позначає границю для    при  . Але права частина останньої нерівності прямує до нуля, якщо  , тобто якщо  . Отже, величина 
     
може бути зроблена як завгодно малою вибором достатньо малих   і  . Це справедливо для будь-якого  , що є початковим елементом точного розв'язку із класу  : але такі елементи щільні в  , так що для будь-якого   з них можна вибрати послідовність   збіжну до   . Тому
                                (12)
Тут два останні члени в правій частині можуть бути зроблені як завгодно малими за допомогою вибору досить великого  , оскільки клас (9) і множина операторів   рівномірно обмежені, а малість першого члена може бути забезпечена за рахунок вибору достатньо малих   і  . Оскільки    - довільний елемент із  , то збіжність встановлена й теорема про еквівалентність доведена.
\end{proof}

Відзначимо в якості природнього наслідку цієї теореми, що для даного початкового елемента   збіжність рівномірна по    на відрізку   у тому розумінні, що обмеження, які потрібно накладати на   і  , щоб зробити (12) нескінченно малим, не залежать ні від вибору   , ні від вибору послідовності  . Ця обставина має велике практичне значення, тому що дозволяє при чисельному інтегруванні знаходити такий крок  , при якому наближений розв'язок виявляється досить точним на всьому відрізку   одночасно. Часто для досягнення потрібної точності   варіюють у процесі обчислень, але існує таке граничне додатнє значення  , нижче якого заходити нема рації.
Викладену теорему Лакса можна дуже просто застосувати до неявного різницевого рівняння для одномірного завдання дифузії

 
Насамперед  покажемо, що розв'язок цього рівняння задовольняє наступному принципу максимуму. Припустимо, що рівняння розглядається в прямокутнику  ,  і що   вибирається рівним   , де    - натуральне число. Тоді максимальне значення  , що досягається величиною    усередині цього прямокутника, не може перевищувати максимального значення  , що досягається початковими й граничними значеннями (тобто  значеннями на відрізках прямих  ). Дійсно, допустимо протилежне, а саме що  , і нехай  - перша внутрішня точка сітки, у якій   (перша в тому розумінні, що індекси    і   мають найменші значення). Тоді в цій точці сітки ліва частина наведеного вище рівняння повинна бути додатна, а права від'ємна, тому що   за припущенням перевершує сусіднє значення   ліворуч і сусіднє значення   знизу й щонайменше   дорівнює сусідньому значенню   праворуч. Отже, наше припущення неправильне, і принцип максимуму встановлений. Очевидно, що ці міркування можна застосувати й до    і тим самим установити, що   обмежені при будь-якому виборі сітки, тобто що розглянуте різницеве рівняння стійке.


Система різницевих рівнянь для задачі (3) має вигляд
 
де   і   – розглянуті раніше різницеві оператори,  а   апроксимує  . Якщо величину   можна зробити як завгодно малою рівномірно за   при   за допомогою вибору  , то останнє різницеве рівняння можна подати у вигляді
     ,  (6)
де   і   обмежені і залежать тільки від   і, можливо, координат, а   можуть бути виражені через  . 
    З (6) випливає  . Якщо виконуються умови теореми Лакса, то остання сума апроксимує інтеграл з (4), звідки випливає збіжність наближеного розв'язку до точного.
