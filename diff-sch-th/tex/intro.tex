\chapter{Що таке теорія різницевих схем?}

\section{Мотивація}

Припустимо, що ми хочемо передбачити, як буде змінюватися стан певної фізичної системи з часом. Що нам для цього потрібно? Зрозуміло, що ми повинні знати, як змінюються ті чи інші фізичні характеристики нашої системи, а також початковий стан цієї системи. \medskip

У рамках цього курсу ми вважаємо, що рівняння чи система рівнянь, яке описує зміну характеристик системи нам вже відоме\footnote{Наприклад, ми вивели його з певних фізичних законів збереження, приклади подібних виведень наводяться у курсі математичної фізики.}. Також вважаємо, що нам задані певні початкові умови, які також виражені рівнянням чи системою рівнянь. \medskip

Може скластися враження, що цього досить, аби знайти точний аналітичний розв'язок (майбутній стан системи як функцію від часу). Це враження хибне, як демонструє наступний приклад:
\begin{align}
    \frac{\partial u}{\partial t} + u \cdot \frac{\partial u}{\partial x} + v \cdot \frac{\partial u}{\partial y} &= - \frac{1}{\rho} \frac{\partial p}{\partial x} + \nu \cdot \left( \frac{\partial^2 u}{\partial x^2} + \frac{\partial^2 u}{\partial y^2} \right) + f_x, \\
    \frac{\partial v}{\partial t} + u \cdot \frac{\partial v}{\partial x} + v \cdot \frac{\partial v}{\partial y} &= - \frac{1}{\rho} \frac{\partial p}{\partial y} + \nu \cdot \left( \frac{\partial^2 v}{\partial x^2} + \frac{\partial^2 v}{\partial y^2} \right) + f_y, \\
    \frac{\partial u}{\partial x} + \frac{\partial v}{\partial y} &= 0.
\end{align}

Тут $t \in \RR_+$ --- часова змінна, $x, y \in \RR$ --- просторові змінні, $u, v: \RR_+ \times \RR^2 \to \RR$ --- компоненти вектора швидкості руху рідини, $f_x, f_y: \RR_+ \times \RR^2 \to \RR$ --- функції, що описують певні сили, які діють на рідину, $\rho: \RR_+ \times \RR^2 \to \RR$ --- густина, $p: \RR_+ \times \RR^2 \to \RR$ --- тиск рідини, $\nu \in \RR_+$ --- так званий коефіцієнт кінематичної в'язкості. \medskip

Це так звані \textit{рівняння Нав'є---Стокса}. Так-так, ті самі, за розв'язання\footnote{А точніше, за доведення існування (чи неіснування) гладких розв'язків зі скінченною кінетичною енергією.} яких математичний інститут Клея дає премію у \$1'000'000. Насправді, ці рівняння трохи простіші ніж справжні загальні рівняння Нав'є---Стокса, і навіть обмежені всього лише двома просторовими змінними, але це не робить їх аналітичне розв'язання простішим. \medskip

Як бачимо, не завжди можна легко знайти аналітичний розв'язок фізичного рівняння, а тому виникає потреба у чисельних методах розв'язування таких задач. У рамках цього курсу ми сфокусуємося на різницевих схемах --- чисельних методах, які засновані на наступному спостереженні:
\begin{equation}
    \frac{\Delta f}{\Delta x} \xrightarrow[\Delta x \to 0]{} \frac{d f}{d x},
\end{equation}
де $df/dx$ --- якась (повна або частинна) похідна функції $f$ за змінною $x$, а $\Delta f/\Delta x$ --- її скінченнорізницевий аналог. Зауважимо, що записана вище збіжність, взагалі кажучи, виконується не завжди, але у колі фізичних процесів які нас цікавитимуть це справді так.

\section{Модельні рівняння}

Для аналізу адекватності того чи іншого чисельного методу потрібно мати \textit{модельне} рівняння, для якого нам відомий точний розв'язок чи принаймні його вигляд. Найпростішими такими рівняннями є
\begin{enumerate}
    \item Лінеаризоване одновимірне рівняння з конвективним і дифузійним членами:
    \begin{equation}
        \frac{\partial \zeta}{\partial t} = -u \cdot \frac{\partial \zeta}{\partial x} + \alpha \cdot \frac{\partial^2 \zeta}{\partial x^2},
    \end{equation}
    де $\zeta$ --- так званий \textit{вихор}, $\zeta = \frac{\partial u}{\partial y} - \frac{\partial v}{\partial x}$ для двовимірних рівнянь.

    \item Рівняння Бюргерса:
    \begin{equation}
        \frac{\partial u}{\partial t} = -u \cdot \frac{\partial u}{\partial x} + \alpha \cdot \frac{\partial^2 u}{\partial x^2}.
    \end{equation}
\end{enumerate}

Обидва ці рівняння зберігають нелінійність рівняння перенесення вихору\footnote{Рівняння перенесення вихору отримується якщо виключити тиск з рівнянь Нав'є---Стокса і має вигляд $\frac{\partial \zeta}{\partial t} = - u \cdot \frac{\partial \zeta}{\partial x} - v \cdot \frac{\partial \zeta}{\partial y} + \nu \cdot \left( \frac{\partial^2 \zeta}{\partial x^2} + \frac{\partial^2 \zeta}{\partial y^2} \right)$.} та рівнянь Нав'є---Стокса, при цьому відомі деякі аналітичні розв'язки рівняння Бюргерса.

\section{Точність апроксимації}

Як ми вже анонсували, для чисельного розв'язання рівнянь математичної фізики (а точніше гідродинаміки) ми будемо заміняти похідні, що фігурують у цих рівняннях на їхні скінченнорізницеві аналоги. Наприклад, похідна $\partial u(x, t) / \partial x$ може бути замінена на $( u(x + \Delta x, t) - u(x, t) ) / \Delta x $. \medskip

Як тільки ми це зробимо виникне логічне запитання: а яка точність такої заміни? Відповідь на це питання можна отримати якщо розкласти всі значення невідомої функції у ряди Тейлора з одним і тим же центром:
\begin{equation}
    \begin{aligned}
        \frac{u(x + \Delta x, t) - u(x, t)}{\Delta x} &= \frac{u(x, t) + \Delta x \cdot \frac{\partial u(x, t)}{\partial x} + O(\Delta x^2) - u(x, t)}{\Delta x} = \\
        &= \frac{\partial u(x, t)}{\partial x} + O(\Delta x).
    \end{aligned}
\end{equation}

Зрозуміло, що можна записати й інші скінченнорізницеві наближення до різних похідних. Найпростішими є наступні:
\begin{enumerate}
    \item $\frac{\partial f}{\partial x} = \frac{f(x + \Delta x) - f(x)}{\Delta x} + O(\Delta x)$ --- \textit{різницева похідна вперед};
    \item $\frac{\partial f}{\partial x} = \frac{f(x) - f(x - \Delta x)}{\Delta x} + O(\Delta x)$ --- \textit{різницева похідна назад};
    \item $\frac{\partial f}{\partial x} = \frac{f(x + \Delta x) - f(x - \Delta x)}{2 \Delta x} + O(\Delta x^2)$ --- \textit{центральна різницева похідна};
    \item $\frac{\partial^2 f}{\partial x^2} = \frac{f(x + \Delta x) - 2 f(x) + f(x - \Delta x)}{2 \Delta x} + O(\Delta x^2)$ --- \textit{друга різницева похідна}.
\end{enumerate}

\section{Сітки та сіткові рівняння}

Подібні заміни призводять до того, що нас по суті перестає цікавити поведінка невідомої функції у всіх-всіх точках. %\footnote{Ще б пак, їх же не просто нескінченна кількість, а континуум, нам має вистачати меншої кількості точок.} 
Це вмотивовує перехід від неперервного рівняння (наприклад, визначеного на $\RR_+ \times \RR^2$) до \textit{сіткового} рівняння, визначеного на певній \textit{сітці}. \medskip

Сітка це певна (скінченна) множина точок, у яких нас цікавитиме значення невідомої функції. Зазвичай беруть так звані рівномірні сітки, хоча вони й не панацея\footnote{Наприклад, якщо розв'язок має суттєво різні градієнти у різних частинах області визначення, то резонним є застосування різноманітних адаптивних методів, детальніше їх висвітлює курс чисельних методів.}. Це означає, що рівняння розглядається на множині
\begin{equation}
    \{n \Delta t\}_{n = 0}^N \times \{i \Delta x\}_{i = 0}^I \times \{j \Delta y\}_{j = 0}^J
\end{equation} 
--- так званій сітці. У такому випадку зазвичай замість $f(i \Delta x, j \Delta y, n \Delta t)$ пишуть $f_{i, j}^n$, що дозволяє значно скоротити рівняння і наблизити їх до синтаксису\footnote{Мова йде про індексацію, тобто звернення до елементу масиву. Справді, \texttt{x[i]}, \texttt{t[n]}, \texttt{u[i, n]} виглядає доволі схоже на математичний запис індексу: $x_i$, $t^n$, $u_i^n$.} одразу кількох основних мов програмування. \medskip

Якщо використати центральні різницеві похідні за часом та простором (як найкращі які у нас поки що є), то у таких позначеннях вищезгадане модельне рівняння Бюргерса матиме вигляд
\begin{equation}
    \label{eq:burgers-unstable}
    \frac{u_i^{n + 1} - u_i^{n - 1}}{\Delta t} = -u_i^n \cdot \frac{u_{i + 1}^n - u_{i - 1}^n}{2 \Delta x} + \alpha \cdot \frac{u_{i + 1}^n - 2 u_i^n + u_{i - 1}^n}{\Delta x^2},
\end{equation}
яке матиме другий порядок точності як за $\Delta t$, так і за $\Delta x$. Якщо ж взяти похідну вперед за часом і центральні похідну за простором, то отримаємо рівняння
\begin{equation}
    \label{eq:burgers-stable}
    \frac{u_i^{n + 1} - u_i^n}{\Delta t} = -u_i^n \cdot \frac{u_{i + 1}^n - u_{i - 1}^n}{2 \Delta x} + \alpha \cdot \frac{u_{i + 1}^n - 2 u_i^n + u_{i - 1}^n}{\Delta x^2},
\end{equation}
яке матиме всього лише перший порядок точності за $\Delta t$ (другий порядок точності за $\Delta x$, зрозуміло, збережеться). \medskip

Втім, якщо це все запрограмувати, то результати будуть доволі неочікуваними. Несподівано виявиться, що перша з цих схем (принаймні для деяких співвідношень на $\Delta t$ і $\Delta x$) даватиме розв'язок, який буде набагато далі від точного, ніж розв'язок, який видасть друга схема. Типовою може бути наступна картина: друга схема видає розв'язок, який осцилює довкола аналітичного, а розв'язок отриманий за першою схемою з часом все далі і далі віддаляється від точного розв'язку. \medskip

Це, звичайно, неприємно, хоча і не дуже несподівано. Справді, ми поки що не доводили ніяких тверджень про зв'язок точності апроксимації диференціального рівняння із точністю наближеного розв'язку. Таким чином, наступний пункт нашої програми --- дослідження точності наближеного розв'язку, або, як заведено казати, \textit{стійкість різницевої схеми}.

\section{Збіжність різницевих схем}

Центральним результатом у питанні дослідження збіжності наближених розв'язків диференціальних рівнянь до точних розв'язків є
\begin{theorem}[Лакса, про еквівалентність]
    Для збіжності наближеного розв'язку коректно поставленої задачі вигляду
    \begin{equation}
        \frac{\partial u(t)}{\partial t} = (A u)(t), \quad u(0) = u_0, \quad 0 \le t \le T
    \end{equation}
    необхідно і достатньо, аби узгоджена скінченнорізницева апроксимація цієї задачі була стійкою.
\end{theorem}

Тут фігурують одразу декілька нових позначень і слів, розберемося по черзі:
\begin{enumerate}
    \item $u(t)$ --- однопараметрична родина елементів певного банахового простору $B$, $A$ --- деякий лінійний оператор, визначений на елементах цього простору, $u_0$ --- певний елемент $B$, який описує початковий стан системи;
    \item родина операторів $C(\Delta t)$ \textit{узгоджено апроксимує} задачу вище, якщо для довільної родини $u(t)$ точних розв'язків цієї задачі з усюди щільними\footnote{Згадуємо курс функціонального аналізу.} в $B$ початковими значеннями виконується умова узгодження:
    \begin{equation}
        \left\| \left( \frac{C(\Delta t) - \text{Id}}{\Delta t} - A \right) u \right\| \xrightarrow[\Delta t \to 0]{} 0,
    \end{equation}
    де норма --- чебишівська на $[0, T]$ ($\|f\| = \max\limits_{0 \le t \le T} |f(t)|$), а $\text{Id}$ --- тотожний оператор;
    \item родина $C(\Delta t)$ називається \textit{стійкою апроксимацією} задачі вище, якщо для деякого $\tau > 0$ родина
    \begin{equation}
        C(\Delta t)^n, \quad n \Delta t \le T, \quad 0 < \Delta t < \tau
    \end{equation}
    рівномірно обмежена\footnote{Згадуємо курс функціонального аналізу.}.
\end{enumerate}

Тут оператори $C(\Delta t)$ і $C(\Delta t)^n$ визначені наступними співвідношеннями: 
\begin{align}
    C(\Delta t): & \quad u^n \mapsto u^{n + 1}, \\
    C(\Delta t)^n: & \quad u^0 \mapsto u^n,
\end{align}
тобто вони задають переходи між часовими рівнями сітки. \medskip

Уважний читач міг зауважити, що теорема Лакса має у певному розумінні обмежену область застосування.

\section{Багатошарові схеми}

А саме, вона передбачає, що ми апроксимуємо наше рівняння так званою \textit{одношаровою} схемою, тобто що $u^{n + 1}$ явно виражається через $u^n$ і тільки через них. Але ж ми вже знаємо щонайменше одну схему для якої це не так (такі схеми називаються багатошаровими). Це \eqref{eq:burgers-unstable}, та сама схема, нестійкість якої схвилювала нас раніше. \medskip

То невже одношарові схеми це панацея, а багатошарові --- якась єресь? Виявляється, що ні, багатошарові схеми мають право на життя, і навіть інколи виявляються кращими за одношарові. Для багатошарових схем існують аналоги теореми Лакса, хоча необхідні умови там і мають складніший вигляд. \medskip

Окрім цього, не завжди зрозуміло звідки брати початкові значення (на більш ніж одному часовому шарі) для багатошарових схем. На це питання також є відповідь, але розгляд цього і суміжних питань виходить за рамки нашого огляду. \medskip

Навіть якщо опустити питання, які виникають при розгляді багатошарових різницевих схем, залишається більш фундаментальна проблема.

\section{Метод фон Неймана}

Теорема Лакса зводить дослідження збіжності численного розв'язку до дослідження стійкості апроксимації, але методів дослідження стійкості у нас немає, а значить необхідно терміново їх отримати. Одним із таких методів є метод фон Неймана. 

\begin{theorem}[критерій стійкості фон Неймана]
    Якщо модуль множника переходу
    \begin{equation}
        G = \epsilon_j^{n + 1} / \epsilon_j^n
    \end{equation}
    менший від одиниці, а початкові умови $u_0(x)$ розвиваються в абсолютно збіжний ряд Фур'є, то точний періодичний розв'язок скінченнорізницевого рівняння може бути поданий\footnote{Тут $a$ --- певна константа, а $M$ --- $I$ з визначення сітки вище, зміна позначень виконана аби уникнути непорозумінь між уявною одиницею та індексом сумування.} у вигляді ряду Фур'є
    \begin{equation}
        u_j^n = \sum_{m = 0}^M e^{a n \Delta t} e^{i k_m j \Delta x},
    \end{equation}
    а апроксимація є стійкою, адже модуль похибки не наростає.
\end{theorem}

Існують аналоги критерію фон Неймана у вищих розмірностях, там фігурує матриця переходу\footnote{Вона описує залежність вектора похибок за просторовими змінними на наступному часовому шарі від вектора похибок за просторовими змінними на цьому часовому шарі.} і її норма замість множника переходу і модуля. Окрім цього іншого вигляду набуває ряд Фур'є, який тепер містить подвійну суму (зовнішню по індексу просторової змінної, а внутрішню --- за \textit{хвильовим числом} $k_m$). \medskip

Спробуємо розібратися на прикладі. Розглянемо одновимірне рівняння теплопровідності:
\begin{equation}
    \frac{\partial u}{\partial t} = \alpha \cdot \frac{\partial^2 u}{\partial x^2}.
\end{equation}

Його скінченнорізницевий аналог має вигляд
\begin{equation}
    u_j^{n + 1} = u_j^n + \frac{\alpha \Delta t}{\Delta x^2} \cdot \Big( u_{j + 1}^n - 2 u_j^n + u_{j - 1}^n \Big).
\end{equation}

Визначаючи похибку $\epsilon_i^n$ як $u(i \Delta x, n \Delta t) - u_i^n$, і враховуючи, що $u$ задовольняє скінченнорізницевому рівнянню точно, отримуємо таке ж скінченнорізницеве рівняння на похибку:
\begin{equation}
    \epsilon_j^{n + 1} = \epsilon_j^n + \frac{\alpha \Delta t}{\Delta x^2} \cdot \Big( \epsilon_{j + 1}^n - 2 \epsilon_j^n + \epsilon_{j - 1}^n \Big).
\end{equation}

Представимо неперервну версію похибки у вигляді її скінченного ряду Фур'є:
\begin{equation}
    \epsilon(x, t) = \sum_{m = 0}^M e^{a t} e^{i k_m x},
\end{equation}

Поведінка такого ряду Фур'є цілком залежить від поведінки його типового члену, тому розглянемо  $\epsilon_m(x, t) = e^{a t} e^{i k_m x}$. Звідси можна знайти
\begin{align}
    \epsilon_j^n &= e^{a t} e^{i k_m x}, \\
    \epsilon_j^{n + 1} &= e^{a (t + \Delta t)} e^{i k_m x}, \\
    \epsilon_{j + 1}^n &= e^{a t} e^{i k_m (x + \Delta x)}, \\
    \epsilon_{j - 1}^n &= e^{a t} e^{i k_m (x - \Delta x)}.
\end{align}

Після підстановки у скінченнорізницеве рівняння, спрощення і приведення подібних, а також пере-позначення $r = \frac{\alpha \Delta t}{\Delta x^2}$ отримаємо
\begin{equation}
    e^{a \Delta t} = 1 + r \cdot \Big( e^{i k_m \Delta x} - 2 + e^{-i k_m \Delta x} \Big),
\end{equation}
або ж, після переходу до тригонометричної  форми комплексного числа
\begin{equation}
    e^{a \Delta t} = 1 - 4 r \sin^2(k_m \Delta x / 2).
\end{equation}

У нашому ж випадку $|G| = |e^{a \Delta t}|$, тому нас цікавить умова
\begin{equation}
    \Big| 1 - 4 r \sin^2(k_m \Delta x / 2) \Big| \le 1,
\end{equation}
% або ж
% \begin{equation}
%     -1 \le 1 - 4 r \sin^2(k_m \Delta x / 2) \le 1,
% \end{equation}
яка виконується для усіх $k_m$ тоді і тільки тоді, коли
\begin{equation}
    r = \frac{\alpha \Delta t}{\Delta x^2} \le \frac{1}{2}.
\end{equation}

Отримана умова і є необхідною умовою стійкості схеми FTCS\footnote{Forward time, central space, це стандартна нотація для найпростіших різницевих схем, спробуйте за аналогією розшифрувати абревіатуру CTCS.} при застосуванні до одновимірного рівняння теплопровідності. \bigskip

\begin{minipage}{.40\textwidth}
    \nothing
\end{minipage}
\begin{minipage}{.57\textwidth}
    Сподіваюся, що цей стислий огляд допоміг читачам
    зрозуміти ідеологію та мету теорії різницевих схем,
    а також побачити загальну картину предмету.
    \smallskip

    \begin{flushright}
        \textit{Нікіта Скибицький, 19 грудня 2019 р.}
    \end{flushright}
\end{minipage}

