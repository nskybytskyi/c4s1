Коли граничні умови були першого роду то до визначення простору $H_A$ додавалися граничні умови $u(0) = u(1) = 0$.
Коли ж граничні умови стали третього роду, то до визначення простору вже нічого не додавалося. 

\begin{remark}
    Справа у тому, що крайові умови діляться на \textit{головні} та \textit{природні} граничні умови.
\end{remark}

$\phi_i$ повинні задовольняти головним похідним, а природні умови ``входять'' до $H_A$.
Для диференціального оператору порядку $2 m$: 
якщо до умови входять лише похідні порядку $< m$ то такі умови головні, а інакше --- природні. 

\begin{example}
    Якщо наш оператор другого порядку, то $m = 1$, і умови першого роду є головними, а умови третього роду
\end{example}

\subsection{Метод найменших квадратів}

\begin{equation}
    \label{eq:3.3.1}
    A u = f
\end{equation}

Нехай існує єдиний розв'язок \eqref{eq:3.3.1}, $A^{-1}$ обмежений, $A: H \to H$ --- лінійний. Для цього методу
\begin{equation}
    \label{eq:3.3.2}
    \Phi(u) = \|A u - f\|^2
\end{equation}

Зрозуміло, що
\begin{equation*}
    \inf_{u \in D(A)} \Phi(u) = 0 = \Phi(u^\star).
\end{equation*}

У нашому випадку
\begin{equation}
    \label{eq:3.3.3}
    \Phi(u) = (A u - f, A u - f) = (A u, A u) - 2 (f, A u) + \|f\|^2
\end{equation}

Згідно загальної теорії маємо 
\begin{align*}
    G(u, v) &= (A u, A v), \\
    \ell(u) &= (f, A u), \\
    c &= (A u, A v).
\end{align*}

\subsubsection{Мінімізаційна послідовність}

\begin{enumerate}
    \item $\{\phi_i\} \subset D(A)$ --- лінійно незалежна, $H$ -- повний;
    \item $H_n = \mathcal{L}(\phi_1, \ldots, \phi_n)$;
    \item Розв'язок шукаємо у вигляді
    \begin{equation}
        \label{eq:3.3.4}
        u_n = \sum_{i = 1}^n c_i \phi_i
    \end{equation}
    \item Отримуємо СЛАР вигляду
    \begin{equation}
        \label{eq:3.3.5}
        \sum_{i = 1}^n c_i (A \phi_i, A \phi_j) = (f, A \phi_j).
    \end{equation}
\end{enumerate}

Зі СЛАР \eqref{eq:3.3.5} бачимо, що СЛАР методу найменших квадратів є СЛАР методу моментів з проекційною системою функцій $\psi_j = A \phi_j$.

\begin{theorem}
    Якщо $\{\phi_i\}$ є $A$-повною (це означає, що $A \phi_i$ повна в $H$) та існує $M$: $\|u\| \le M \|A u\|$ 
    ($A^{-1}$ обмежений) то мінізаційна послідовність $u_n$ буде збігатися до $u$ в нормі простору $H$: 
    $\{u_n\} \xrightarrow[n \to \infty]{} u$.
\end{theorem}

\begin{remark}
    Варто вибирати $\{\phi_i\}$ ортогональними, щоб СЛАР мала мале число обумовленості.
\end{remark}

\section{Сіткові (дискретні/різницеві) методи}

\subsection{Загальні поняття методу сіток}

Розгялдаємо рівняння
\begin{equation}
    \label{eq:4.1.1}
    A u = f,
\end{equation}
де $A: B_1 \to B_2$ ($B_1$, $B_2$ --- банахові простори, $D(A) \subseteq B_1$, $D(A) \subseteq B_2$. \medskip

Головна ідея методу сіток полягає у введенні просторів $B_{1,h}$ та $B_{2,h}$, 
які залежать від певного $\vec h = (h_1, \ldots, h_n)$. 
Далі замінюємо $A$ оператором $A_h: B_{1,h} \to B_{2,h}$. Задачу \eqref{eq:4.1.1} замінюємо задачею
\begin{equation}
    \label{eq:4.1.2}
    A_h y_h = \phi_h.
\end{equation}

Суть у тому, що $B_{1,h}$ та $B_{2,h}$ --- простори скінченновимірні, наприклад сіткові.

% Виникає три питання: як перейти від \eqref{eq:4.1.1} до \eqref{eq:4.1.2}, як оцынити похи

\begin{definition}
    Задача \eqref{eq:4.1.1} \textit{коректно поставлена}, якщо:
    \begin{enumerate}
        \item $\exists!$ розв'язок \eqref{eq:4.1.1} $\forall f \in R(A)$;
        \item цей розв'язок стійки, тобто $\|\tilde u - u\|_1 \le M \|\tilde f - f\|_2$, де $M$ стала;
    \end{enumerate}
\end{definition}

\begin{definition}
    Задача \eqref{eq:4.1.2} \textit{коректно поставлена}, якщо:
    \begin{enumerate}
        \item $\exists!$ розв'язок \eqref{eq:4.1.2} $\forall f \in R(A)$;
        \item цей розв'язок стійки, тобто $\|\tilde y_h - y_h\|_{1,h} \le M_1 \|\tilde \phi_h - \phi_h\|_{2,h}$, 
        де $M_1 > 0$ --- (можливо інша) стала;
    \end{enumerate}
\end{definition}

Розглянемо оператори проектування $P_{1, h}: B_1 \to B_{1, h}: P_{1, h} u = u_h$ та $P_{2, h}: B_2 \to B_{2, h}: P_{2, h} f = f_h$. 
Окрім цього, хочеться мати узгоджені норми в цих просторах, тобто $\lim_{|h| \to 0} \|P_{1,h} u\|_{1,h} = \|u\|_1$ і $\lim_{|h| \to 0} \|P_{2,h} f\|_{2,h} = \|f\|_2$

\begin{definition}
    Функція 
    \begin{equation}
        \label{eq:4.1.3}
        z_h = y_h - P_{1, h} u = y_h - u_h
    \end{equation}
     --- похибка різницевої задачі \eqref{eq:4.1.2}, визначена на просторі $B_{1,h}$.
\end{definition}

\begin{definition}
    Будемо говорити, що розв'язок задачі \eqref{eq:4.1.2} \textit{збігається} до розв'язку задачі \eqref{eq:4.1.1} (або ж, що різницева схема збіжна), якщо $\lim_{|h| \to 0} \|z_n\|_{1,h} \to 0$.
\end{definition}

\begin{definition}
    Будемо говорити, що розв'язок задачі \eqref{eq:4.1.2} \textit{збігається} до розв'язку задачі \eqref{eq:4.1.1} \textit{з порядком $m$}, якщо 
    \begin{equation}
        \label{eq:4.1.4}
        \|z_n\|_{1,h} = O(|h|^m).
    \end{equation}
\end{definition}
\begin{remark}
    Тут $O(|h|^m) = C \cdot |h|^m$, де $C > 0$ --- незалежна від $|h|$ константа.
\end{remark}


З формули \eqref{eq:4.1.3} маємо 
\begin{equation*}
    y_h = z_h + u_n.
\end{equation*}

Позначимо
\begin{equation}
    \label{eq:4.1.5}
    A_h z_h = \psi_h,
\end{equation}
тоді, підставляючи в \eqref{eq:4.1.2} маємо
\begin{equation}
    \label{eq:4.1.6}
    \psi_h = \phi_h - A_h u_h.
\end{equation}

\begin{definition}
    Величина $\psi_h$ визначена є \textit{похибкою апроксимації} (нев'язкою) різницевої схеми \eqref{eq:4.1.2} на розв'язку $y$.
\end{definition}

\begin{equation}
    \label{eq:4.1.7}
    \psi_h = \phi_h - A_h u_h - P_{2, h} (f - A u) = (\phi_h - P_{2, h} f) - (A_h u_h - P_{2, h} A u) = \psi_h^f - \psi_h^A.
\end{equation}

Звідси бачимо, що похибка апроксимації схеми складаєтся з похибки апроксимації правих частин:
\begin{equation}
    \label{eq:4.1.8}
    \psi_h^f = \phi_h + P_{2, h} f = \phi_h - f_h,
\end{equation}
та похибки апроксимації диференціального оператору $A$ різницеми оператором $A_h$:
\begin{equation}
    \label{eq:4.1.9}
    \psi_h^A = A_h u_n - P_{2, h} A u.
\end{equation}

\begin{lemma}[Лакса-Філіпова]
    Нехай різницева задача \eqref{eq:4.1.2} апроксимує задачу \eqref{eq:4.1.1} та є коректно поставленою. Тоді розв'язок задачі \eqref{eq:4.1.2} буде збігатися до розв'язку задачі \eqref{eq:4.1.1}, причому порядок збіжності буде таким же, як і порядок апроксимації.
\end{lemma}
\begin{proof}
    Підставимо
    \begin{equation}
        z_h = y_h - u_h
    \end{equation}
    в \eqref{eq:4.1.2}. Тоді для похибки отримуємо задачу \eqref{eq:4.1.6}. За умовою лемми задача є коректно поставленою, а тому і стійкої. З умови стійкості випливає, що
    \begin{equation}
        \label{eq:4.1.10}
        \|z_h\|_{1, h} \le M \|\psi_h\|_{2,h}.
    \end{equation}

    Звідси і випливає, що 
    \begin{equation}
        \label{eq:4.1.11}
        \lim_{|h| \to 0} \|z_h\|_{1, h} \le M \lim_{|h| \to 0} \|\psi_h\|_{2, h} = 0.
    \end{equation}

    Причому,
    \begin{equation}
        \label{eq:4.1.12}
        \lim_{|h| \to 0} \|z_h\|_{1, h} \le M \cdot O(|h|^m) = O(|h|^m).
    \end{equation}
    \stepcounter{equation}
\end{proof}

\begin{remark}
    $z_h$ --- стійкість і $\psi_h$. $\psi_h$ --- $\phi_h \sim f_h$ і $A_h \sim A$.
    Тому лему ще називають ``апроксимація + стійкість = збіжність''.
\end{remark}

\begin{definition}
    Будемо говорити, що $A_h$ \textit{наближає} $A$ якщо
    \begin{equation}
        \label{eq:4.1.14}
        \lim_{|h| \to 0} \|A_h u_h - P_{2, h} A u\|_{2, h} = 0.
    \end{equation}
\end{definition}

\begin{definition}
    Будемо говорити, що $A_h$ \textit{наближає} $A$ \textit{з порядком $m$} якщо
    \begin{equation}
        \label{eq:4.1.15}
        \|A_h u_h - P_{2, h} A u\|_{2, h} = O(|h|^m).
    \end{equation}
\end{definition}

\begin{definition}
    Будемо говорити, що $\phi_h$ \textit{наближає} $f$ якщо
    \begin{equation}
        \label{eq:4.1.16}
        \lim_{|h| \to 0} \|\phi_h - P_{2, h} f\|_{2, h} = 0.
    \end{equation}
\end{definition}

\begin{definition}
    Будемо говорити, що $\phi_h$ \textit{наближає} $f$ \textit{з порядком $m$} якщо
    \begin{equation}
        \label{eq:4.1.17}
        \|\phi_h - P_{2, h} f\|_{2, h} = O(|h|^m).
    \end{equation}
\end{definition}
