\title[Дискретно сумовні СЛАР]
{\S1.2. Дискретно сумовні системи \\ лінійних алгебраїчних рівнянь}

%% first slide
\begin{frame}
    \titlepage
\end{frame}
%% first slide

%% slide 01
\begin{mframe}{Постановка задачі}
    Розглянемо узагальнення розв'язку СЛАР $A x = b$ з попередньої лекції на
    дискретно сумовну СЛАР:
    \begin{equation}
        \label{eq:1.9}
        \Sum_{i = 1}^N A_i x_i = b,
    \end{equation}
    де $A_i \in \mathbb{R}^{m \times n}$ --- відомі матриці,
    $x_i \in \mathbb{R}^n$ --- невідомі вектори, $b \in \mathbb{R}^m$ ---
    відомий вектор. \medskip
    
    Як і звичайні СЛАР, дискретно сумовна СЛАР може мати або не мати обернення
    (однозначне або множину обернень). В останньому випадку, як і для звичайних
    СЛАР, обмежимося середньоквадратичним наближенням до такого обернення.
\end{mframe}
%% slide 01

%% slide 02
\begin{mframe}{Множина розв'язків}
    Введемо множину
    \begin{equation}
        \label{eq:1.10}
        \Omega_x = \left\{ (x_1, \ldots, x_N): 
        \left\| \Sum_{i = 1}^N A_i x_i - b \right\|^2 = \min_{z_1, \ldots, z_N}
        \left\| \Sum_{i = 1}^N A_i z_i - b \right\|^2 \right\}.
    \end{equation}
    
    Можна показати, що
    \begin{equation}
        \label{eq:1.12}
        x_i \in \Big\{ A_i^\intercal P_1^+ b + v_i - A_i^\intercal P_1^+ A_v 
        \Big| v_i \in \mathbb{R}^n \Big\},
    \end{equation}
    де
    \begin{equation*}
        P_1 = \Sum_{i = 1}^N A_i A_i^\intercal, \quad A_v = 
        \Sum_{i = 1}^N A_i v_i.
    \end{equation*}
\end{mframe}
%% slide 02

%% slide 03
\begin{mframe}{Виділення однозначного розв'язку}
    За неоднозначності $\Omega_x$ виділимо з неї вектори $\bar x_i$ такі, що
    \begin{equation}
        \label{eq:1.11}
        \bar x_i = \Argmin_{(x_1, \ldots, x_N) \in \Omega_x} \|x_i\|^2.
    \end{equation}
    
    Можна показати, що
    \begin{equation}
        \label{eq:1.14}
        \bar x_i = A_i^\intercal P_1^+ b.
    \end{equation}
\end{mframe}
%% slide 03

%% slide 04
\begin{mframe}{Однозначність і точність розв'язку}
    Розв'язки $\bar x_i$ СЛАР \eqref{eq:1.9} будуть однозначними, якщо
    \begin{equation}
        \label{eq:1.13}
        \det \begin{bmatrix}
            A_1^\intercal A_1 & A_1^\intercal A_2 & \cdots & A_1^\intercal A_N\\
            A_2^\intercal A_1 & A_2^\intercal A_2 & \cdots & A_2^\intercal A_N\\
            \vdots & \vdots & \ddots & \vdots \\
            A_N^\intercal A_1 & A_N^\intercal A_2 & \cdots & A_N^\intercal A_N
        \end{bmatrix} > 0.
    \end{equation}

    Точність розв'язку оцінюється величиною
    \begin{equation}
        \label{eq:1.15}
        \varepsilon^2 = \min_{(x_1, \ldots, x_N) \in \Omega_x}
        \left\| \Sum_{i = 1}^N A_i x_i - b \right\|^2 =
        b^\intercal b - b^\intercal P_1 P_1^+ b.
    \end{equation}
\end{mframe}
%% slide 04

%% slide 05
\begin{mframe}{У напрямку інтегральної задачі}
    Розглянемо задачу
    \begin{equation}
        \label{eq:1.16}
        \Sum_{i = 1}^N A(t_i) x(t_i) = b,
    \end{equation}
    де $A: \mathbb{R} \to \mathbb{R}^{m \times n}$ --- відома матрично-значна
    функція скалярного аргументу, $x: \mathbb{R} \to \mathbb{R}^n$ --- невідома
    вектор-функція скалярного аргументу, $b \in \mathbb{R}^m$ --- відомий
    вектор. Моменти часу $t_i$ цілком конкретні і фіксовані. \medskip
    
    Цілком очевидно, що вона еквівалентна попередній задачі, тому просто
    наведемо для неї аналогічні результати.
\end{mframe}
%% slide 05

%% slide 06
\begin{mframe}{Множина розв'язків}
    Введемо множину
    \begin{multline}
        \label{eq:1.17}
        \Omega_x = \left\{ x(t_i), i = \overline{1,N}: 
        \left\| \Sum_{i = 1}^N A(t_i) x(t_i) - b \right\|^2 \right. = \\ 
        = \left. \min_{z(t_i), i = \overline{1,N}} 
        \left\| \Sum_{i = 1}^N A(t_i) z(t_i) - b \right\|^2 \right\}.
    \end{multline}
    
    Можна показати, що
    \begin{equation}
        \label{eq:1.19}
        x(t_i) \in \Big\{ A^\intercal(t_i) P_1^+ b + v(t_i) - A^\intercal(t_i) 
        P_1^+ A_v \Big| v: \mathbb{R} \to \mathbb{R}^n \Big\},
    \end{equation}
    де
    \begin{equation*}
        P_1 = \Sum_{i = 1}^N A(t_i) A^\intercal(t_i), \quad
        A_v = \Sum_{i = 1}^N A(t_i) v(t_i).
    \end{equation*}
\end{mframe}
%% slide 06

%% slide 07
\begin{mframe}{Виділення однозначного розв'язку}
    За неоднозначності $\Omega_x$ виділимо з неї вектори $\bar x(t_i)$ такі, що
    \begin{equation}
        \label{eq:1.18}
        \bar x(t_i) = 
        \Argmin_{(x(t_i), i = \overline{1, N}) \in \Omega_x} \|x(t_i)\|^2.
    \end{equation}
    
    Можна показати, що
    \begin{equation}
        \label{eq:1.20}
        \bar x(t_i) = A^\intercal(t_i) P_1^+ b.
    \end{equation}
\end{mframe}
%% slide 07

%% slide 08
\begin{mframe}{Однозначність і точність розв'язку}
    Розв'язки $\bar x(t_i)$ СЛАР \eqref{eq:1.16} будуть однозначними, якщо
    \begin{equation}
        \label{eq:1.21}
        \det \begin{bmatrix}
            A^\intercal(t_1) A(t_1) & A^\intercal(t_1) A(t_2) & \cdots & 
                A^\intercal(t_1) A(t_N) \\
            A^\intercal(t_2) A(t_1) & A^\intercal(t_2) A(t_2) & \cdots & 
                A^\intercal(t_2) A(t_N) \\
            \vdots & \vdots & \ddots & \vdots \\
            A^\intercal(t_N) A(t_1) & A^\intercal(t_N) A(t_2) & \cdots & 
                A^\intercal(t_N) A(t_N)
        \end{bmatrix} > 0.
    \end{equation}

    Точність розв'язку оцінюється величиною
    \begin{equation}
        \label{eq:1.22}
        \varepsilon^2 = \min_{(x(t_i), i = \overline{1, N}) \in \Omega_x} 
        \left\| \Sum_{i = 1}^N A(t_i) x(t_i) - b \right\|^2 =
        b^\intercal b - b^\intercal P_1 P_1^+ b.
    \end{equation}
\end{mframe}
%% slide 08
